% Options for packages loaded elsewhere
\PassOptionsToPackage{unicode}{hyperref}
\PassOptionsToPackage{hyphens}{url}
\PassOptionsToPackage{dvipsnames,svgnames,x11names}{xcolor}
%
\documentclass[
  letterpaper,
  DIV=11,
  numbers=noendperiod]{scrreprt}

\usepackage{amsmath,amssymb}
\usepackage{iftex}
\ifPDFTeX
  \usepackage[T1]{fontenc}
  \usepackage[utf8]{inputenc}
  \usepackage{textcomp} % provide euro and other symbols
\else % if luatex or xetex
  \usepackage{unicode-math}
  \defaultfontfeatures{Scale=MatchLowercase}
  \defaultfontfeatures[\rmfamily]{Ligatures=TeX,Scale=1}
\fi
\usepackage{lmodern}
\ifPDFTeX\else  
    % xetex/luatex font selection
\fi
% Use upquote if available, for straight quotes in verbatim environments
\IfFileExists{upquote.sty}{\usepackage{upquote}}{}
\IfFileExists{microtype.sty}{% use microtype if available
  \usepackage[]{microtype}
  \UseMicrotypeSet[protrusion]{basicmath} % disable protrusion for tt fonts
}{}
\makeatletter
\@ifundefined{KOMAClassName}{% if non-KOMA class
  \IfFileExists{parskip.sty}{%
    \usepackage{parskip}
  }{% else
    \setlength{\parindent}{0pt}
    \setlength{\parskip}{6pt plus 2pt minus 1pt}}
}{% if KOMA class
  \KOMAoptions{parskip=half}}
\makeatother
\usepackage{xcolor}
\setlength{\emergencystretch}{3em} % prevent overfull lines
\setcounter{secnumdepth}{5}
% Make \paragraph and \subparagraph free-standing
\makeatletter
\ifx\paragraph\undefined\else
  \let\oldparagraph\paragraph
  \renewcommand{\paragraph}{
    \@ifstar
      \xxxParagraphStar
      \xxxParagraphNoStar
  }
  \newcommand{\xxxParagraphStar}[1]{\oldparagraph*{#1}\mbox{}}
  \newcommand{\xxxParagraphNoStar}[1]{\oldparagraph{#1}\mbox{}}
\fi
\ifx\subparagraph\undefined\else
  \let\oldsubparagraph\subparagraph
  \renewcommand{\subparagraph}{
    \@ifstar
      \xxxSubParagraphStar
      \xxxSubParagraphNoStar
  }
  \newcommand{\xxxSubParagraphStar}[1]{\oldsubparagraph*{#1}\mbox{}}
  \newcommand{\xxxSubParagraphNoStar}[1]{\oldsubparagraph{#1}\mbox{}}
\fi
\makeatother


\providecommand{\tightlist}{%
  \setlength{\itemsep}{0pt}\setlength{\parskip}{0pt}}\usepackage{longtable,booktabs,array}
\usepackage{calc} % for calculating minipage widths
% Correct order of tables after \paragraph or \subparagraph
\usepackage{etoolbox}
\makeatletter
\patchcmd\longtable{\par}{\if@noskipsec\mbox{}\fi\par}{}{}
\makeatother
% Allow footnotes in longtable head/foot
\IfFileExists{footnotehyper.sty}{\usepackage{footnotehyper}}{\usepackage{footnote}}
\makesavenoteenv{longtable}
\usepackage{graphicx}
\makeatletter
\def\maxwidth{\ifdim\Gin@nat@width>\linewidth\linewidth\else\Gin@nat@width\fi}
\def\maxheight{\ifdim\Gin@nat@height>\textheight\textheight\else\Gin@nat@height\fi}
\makeatother
% Scale images if necessary, so that they will not overflow the page
% margins by default, and it is still possible to overwrite the defaults
% using explicit options in \includegraphics[width, height, ...]{}
\setkeys{Gin}{width=\maxwidth,height=\maxheight,keepaspectratio}
% Set default figure placement to htbp
\makeatletter
\def\fps@figure{htbp}
\makeatother
% definitions for citeproc citations
\NewDocumentCommand\citeproctext{}{}
\NewDocumentCommand\citeproc{mm}{%
  \begingroup\def\citeproctext{#2}\cite{#1}\endgroup}
\makeatletter
 % allow citations to break across lines
 \let\@cite@ofmt\@firstofone
 % avoid brackets around text for \cite:
 \def\@biblabel#1{}
 \def\@cite#1#2{{#1\if@tempswa , #2\fi}}
\makeatother
\newlength{\cslhangindent}
\setlength{\cslhangindent}{1.5em}
\newlength{\csllabelwidth}
\setlength{\csllabelwidth}{3em}
\newenvironment{CSLReferences}[2] % #1 hanging-indent, #2 entry-spacing
 {\begin{list}{}{%
  \setlength{\itemindent}{0pt}
  \setlength{\leftmargin}{0pt}
  \setlength{\parsep}{0pt}
  % turn on hanging indent if param 1 is 1
  \ifodd #1
   \setlength{\leftmargin}{\cslhangindent}
   \setlength{\itemindent}{-1\cslhangindent}
  \fi
  % set entry spacing
  \setlength{\itemsep}{#2\baselineskip}}}
 {\end{list}}
\usepackage{calc}
\newcommand{\CSLBlock}[1]{\hfill\break\parbox[t]{\linewidth}{\strut\ignorespaces#1\strut}}
\newcommand{\CSLLeftMargin}[1]{\parbox[t]{\csllabelwidth}{\strut#1\strut}}
\newcommand{\CSLRightInline}[1]{\parbox[t]{\linewidth - \csllabelwidth}{\strut#1\strut}}
\newcommand{\CSLIndent}[1]{\hspace{\cslhangindent}#1}

\KOMAoption{captions}{tableheading}
\makeatletter
\@ifpackageloaded{bookmark}{}{\usepackage{bookmark}}
\makeatother
\makeatletter
\@ifpackageloaded{caption}{}{\usepackage{caption}}
\AtBeginDocument{%
\ifdefined\contentsname
  \renewcommand*\contentsname{Table of contents}
\else
  \newcommand\contentsname{Table of contents}
\fi
\ifdefined\listfigurename
  \renewcommand*\listfigurename{List of Figures}
\else
  \newcommand\listfigurename{List of Figures}
\fi
\ifdefined\listtablename
  \renewcommand*\listtablename{List of Tables}
\else
  \newcommand\listtablename{List of Tables}
\fi
\ifdefined\figurename
  \renewcommand*\figurename{Figure}
\else
  \newcommand\figurename{Figure}
\fi
\ifdefined\tablename
  \renewcommand*\tablename{Table}
\else
  \newcommand\tablename{Table}
\fi
}
\@ifpackageloaded{float}{}{\usepackage{float}}
\floatstyle{ruled}
\@ifundefined{c@chapter}{\newfloat{codelisting}{h}{lop}}{\newfloat{codelisting}{h}{lop}[chapter]}
\floatname{codelisting}{Listing}
\newcommand*\listoflistings{\listof{codelisting}{List of Listings}}
\makeatother
\makeatletter
\makeatother
\makeatletter
\@ifpackageloaded{caption}{}{\usepackage{caption}}
\@ifpackageloaded{subcaption}{}{\usepackage{subcaption}}
\makeatother

\ifLuaTeX
  \usepackage{selnolig}  % disable illegal ligatures
\fi
\usepackage{bookmark}

\IfFileExists{xurl.sty}{\usepackage{xurl}}{} % add URL line breaks if available
\urlstyle{same} % disable monospaced font for URLs
\hypersetup{
  pdftitle={Konsep Pembelajaran Kokreasi Nilai},
  pdfauthor={Armein Z. R. Langi},
  colorlinks=true,
  linkcolor={blue},
  filecolor={Maroon},
  citecolor={Blue},
  urlcolor={Blue},
  pdfcreator={LaTeX via pandoc}}


\title{Konsep Pembelajaran Kokreasi Nilai}
\author{Armein Z. R. Langi}
\date{2025-08-07}

\begin{document}
\maketitle

\renewcommand*\contentsname{Table of contents}
{
\hypersetup{linkcolor=}
\setcounter{tocdepth}{2}
\tableofcontents
}

\bookmarksetup{startatroot}

\chapter{Pengantar}\label{pengantar}

This is an excellent pedagogical approach for a signals and systems
course, as it encourages students to move beyond rote memorization and
develop a deeper, more integrated understanding of the subject. Your
idea of creating two types of knowledge maps is well-supported by
research on learning and problem-solving. Here's a breakdown of how you
can implement this for your signals and systems course:

\section{Part 1: The Primitive Knowledge
Map}\label{part-1-the-primitive-knowledge-map}

This map is about the ``atomic'' concepts of the course---the
fundamental building blocks. It's a static representation of the core
body of knowledge.

\begin{enumerate}
\def\labelenumi{\arabic{enumi}.}
\tightlist
\item
  Identify Core Concepts and Principles:
\end{enumerate}

Begin by outlining all the essential topics in your course. This should
be a comprehensive list of all the ``nodes'' in your map. Examples for
signals and systems would include:

\begin{itemize}
\item
  \textbf{Signals:} Continuous-time (CT) vs.~discrete-time (DT),
  periodic vs.~aperiodic, energy vs.~power signals, common signals (unit
  step, impulse, ramp, sinusoids).
\item
  \textbf{Systems:} CT vs.~DT, linearity, time-invariance, causality,
  stability.
\item
  \textbf{Transformations:} Fourier Series, Fourier Transform (CTFT and
  DTFT), Laplace Transform, Z-Transform.
\item
  \textbf{Analysis:} Convolution, transfer functions, frequency
  response, pole-zero plots, sampling theorem.
\item
  \textbf{Applications:} Filtering (low-pass, high-pass), modulation.
\end{itemize}

\begin{enumerate}
\def\labelenumi{\arabic{enumi}.}
\setcounter{enumi}{1}
\tightlist
\item
  Define the Relationships (``Edges''):
\end{enumerate}

This is the most crucial part of the primitive map. The ``edges''
connecting the concepts are the relationships between them. These
relationships are the ``vehicles'' you mentioned. Labeling these
connections is key to making the map useful.

\begin{itemize}
\item
  \textbf{Cause-and-Effect:} ``A linear system \emph{is defined by}
  additivity and homogeneity.''
\item
  \textbf{Example/Instance:} ``The unit step function \emph{is an
  example of} a CT signal.''
\item
  \textbf{Mathematical Equivalence:} ``The transfer function H(s)
  \emph{is the Laplace transform of} the impulse response h(t).''
\item
  \textbf{Generalization/Special Case:} ``The Fourier Transform \emph{is
  a generalization of} the Fourier Series.''
\item
  \textbf{Analysis Tool:} ``The pole-zero plot \emph{is used to analyze}
  system stability.''
\end{itemize}

\begin{enumerate}
\def\labelenumi{\arabic{enumi}.}
\setcounter{enumi}{2}
\tightlist
\item
  Visual Representation:
\end{enumerate}

You can use a variety of formats, from simple mind maps to more
structured concept maps. Encourage students to create their own. The act
of building the map is a learning experience in itself.

\begin{itemize}
\item
  \textbf{Mind Map:} A central concept (e.g., ``Signals and Systems'')
  with branches radiating out for major topics.
\item
  \textbf{Concept Map:} A hierarchical network with nodes and labeled
  links. This is often more powerful for technical subjects as it forces
  students to explicitly state the relationship between concepts.
\item
  \textbf{Collaborative Whiteboard:} Use a tool like Miro or a physical
  whiteboard in class to build a shared map with the students.
\end{itemize}

\section{Part 2: The Problem-Solving Knowledge
Map}\label{part-2-the-problem-solving-knowledge-map}

This map is dynamic and process-oriented. It's a collection of
``routes'' or problem-solving strategies, each built from a sequence of
primitive knowledge. The ``vehicles'' here are the specific mathematical
operations and concepts used to move from one state of knowledge to
another.

\begin{enumerate}
\def\labelenumi{\arabic{enumi}.}
\tightlist
\item
  Frame Problems as Gaps:
\end{enumerate}

As you suggested, every problem starts with what's known and ends with
what's asked.

\begin{itemize}
\item
  \textbf{Starting Point (Knowns):} The given information in the
  problem. For example, ``A system is described by a differential
  equation.'' or ``The input signal is a sum of sinusoids.''
\item
  \textbf{Ending Point (Asked):} The desired output or conclusion. For
  example, ``Find the output signal y(t),'' or ``Determine if the system
  is stable.''
\end{itemize}

\begin{enumerate}
\def\labelenumi{\arabic{enumi}.}
\setcounter{enumi}{1}
\tightlist
\item
  Develop ``Routes'' and ``Vehicles'':
\end{enumerate}

The routes are the sequence of steps, and the vehicles are the specific
operations (e.g., convolution, Fourier transform, etc.) that connect the
knowns to the unknowns. This is where you connect the primitive
knowledge to practical application.

\begin{itemize}
\item
  \textbf{Route 1: Time Domain Analysis}

  \begin{itemize}
  \item
    \textbf{Starting Point:} System impulse response h(t) and input
    signal x(t).
  \item
    \textbf{Vehicle:} Convolution integral:
    y(t)=∫−∞∞\hspace{0pt}x(τ)h(t−τ)dτ.
  \item
    \textbf{Ending Point:} System output y(t).
  \end{itemize}
\item
  \textbf{Route 2: Frequency Domain Analysis}

  \begin{itemize}
  \item
    \textbf{Starting Point:} System transfer function H(s) and input
    signal X(s).
  \item
    \textbf{Vehicle 1:} Multiplication in the frequency domain:
    Y(s)=X(s)H(s).
  \item
    \textbf{Vehicle 2:} Inverse Laplace Transform: y(t)=L−1\{Y(s)\}.
  \item
    \textbf{Ending Point:} System output y(t).
  \end{itemize}
\item
  \textbf{Route 3: Stability Analysis}

  \begin{itemize}
  \item
    \textbf{Starting Point:} System transfer function H(s).
  \item
    \textbf{Vehicle:} Find poles of H(s) by solving for roots of the
    denominator polynomial.
  \item
    \textbf{Ending Point:} Conclusion on stability (e.g., ``System is
    stable if all poles are in the left-half of the s-plane'').
  \end{itemize}
\end{itemize}

\textbf{3. Teaching the Approach:}

\begin{itemize}
\item
  \textbf{Model the Process:} During lectures, explicitly ``walk the
  map.'' When you introduce a problem, identify the starting and ending
  points. Then, show the students the possible routes (e.g., time domain
  vs.~frequency domain) and the vehicles needed for each step.
\item
  \textbf{Scaffolded Practice:} Start with simple problems where there
  is a clear, single best route. Gradually introduce more complex
  problems where students have to choose between multiple routes.
\item
  \textbf{``Problem-Solving Checklists'' (mini-maps):} For common
  problem types, have students create small ``checklists'' or mini-maps.
  For example, a checklist for ``determining system properties'' would
  be: ``Test for Linearity (additivity, homogeneity) -\textgreater{}
  Test for Time-Invariance -\textgreater{} Test for Causality
  -\textgreater{} Test for Stability.''
\item
  \textbf{Reflective Assignment:} Ask students to solve a problem and
  then, as a follow-up, write a short paragraph explaining their chosen
  route and why it was the best choice. This makes them metacognitive
  about their problem-solving process.
\end{itemize}

\section{Summary: Your Two Maps}\label{summary-your-two-maps}

\begin{itemize}
\item
  \textbf{Primitive Knowledge Map:} This is the \textbf{encyclopedia}.
  It contains the definitions, properties, and fundamental relationships
  of all the concepts in the course. It's the reference material.
\item
  \textbf{Problem-Solving Knowledge Map:} This is the \textbf{GPS}. It's
  the dynamic tool that guides a student through a problem. It helps
  them identify their current location (the knowns), their destination
  (the unknowns), and the best path to get there using the ``vehicles''
  of the course.
\end{itemize}

By teaching students to create and use both types of maps, you empower
them to not only understand the material but also to think like
engineers---strategically and systematically.

\bookmarksetup{startatroot}

\chapter{Laporan Ahli: Konsep Dasar Sinyal dan Sistem: Peta Pengetahuan
dan Kerangka Pemecahan
Masalah}\label{laporan-ahli-konsep-dasar-sinyal-dan-sistem-peta-pengetahuan-dan-kerangka-pemecahan-masalah}

\section{I. Pendahuluan: Fondasi Sinyal dan
Sistem}\label{i.-pendahuluan-fondasi-sinyal-dan-sistem}

Analisis sinyal dan sistem merupakan pilar fundamental dalam berbagai
disiplin ilmu teknik, termasuk teknik elektro, telekomunikasi, kontrol,
dan pemrosesan citra. Disiplin ini menyediakan kerangka kerja matematis
untuk memahami, memodelkan, dan memanipulasi fenomena yang bervariasi
terhadap waktu atau ruang. Inti dari konsep ini adalah interaksi antara
sinyal input, sistem atau medium yang memprosesnya, dan sinyal output
yang dihasilkan. Kemampuan untuk menganalisis dan memecahkan
permasalahan yang melibatkan ketiga elemen ini sangat krusial dalam
rekayasa sistem modern.

\section{A. Konsep Dasar Sinyal dan
Sistem}\label{a.-konsep-dasar-sinyal-dan-sistem}

Sinyal dan sistem adalah dua entitas yang saling terkait erat dalam
studi teknik. Pemahaman yang mendalam tentang keduanya merupakan
prasyarat untuk merancang dan menganalisis sistem yang kompleks.

\subsection{Definisi Sinyal}\label{definisi-sinyal}

Sinyal didefinisikan sebagai representasi dari suatu fenomena fisik yang
membawa informasi dan bervariasi terhadap satu atau lebih variabel
independen, seperti waktu, ruang, atau variabel lainnya.1 Secara
matematis, sinyal sering dinyatakan sebagai fungsi, misalnya

x(t) untuk sinyal waktu kontinu atau x{[}n{]} untuk sinyal waktu
diskrit. Contoh umum sinyal meliputi gelombang suara, yang
direpresentasikan sebagai perubahan tekanan akustik terhadap waktu;
tegangan atau arus listrik dalam suatu rangkaian; atau citra digital,
yang dapat digambarkan sebagai fungsi kecerahan dua variabel spasial
(koordinat piksel).2 Sinyal dapat memiliki nilai real atau skalar, dan
karakteristiknya, seperti periodisitas, energi, daya, serta sifat genap
atau ganjil, menjadi fokus utama dalam analisisnya.3

\subsection{Definisi Sistem}\label{definisi-sistem}

Sistem adalah entitas atau proses yang berinteraksi dengan sinyal input
untuk menghasilkan sinyal output atau respons.1 Dalam konteks matematis,
sistem dapat dipandang sebagai transformasi atau pemetaan yang mengubah
sinyal input

x menjadi sinyal output y, sering dinotasikan sebagai y=H\{x\}, di mana
H adalah operator yang merepresentasikan aturan transformasi.7 Sistem
dapat berupa perangkat fisik (seperti rangkaian elektronik atau alat
kesehatan) atau algoritma pemrosesan data.1

\subsection{Hubungan Fundamental
Input-Sistem-Output}\label{hubungan-fundamental-input-sistem-output}

Konsep dasar sinyal dan sistem berpusat pada tiga elemen utama yang
saling terkait: sinyal input, sistem atau medium, dan sinyal output yang
dihasilkan sebagai respons terhadap input {[}Query{]}. Sistem memproses
sinyal input untuk menghasilkan sinyal output. Dalam analisis sistem,
terdapat tiga permasalahan dasar yang sering dijumpai: (1) mencari
sinyal output ketika sinyal input dan karakteristik sistem diketahui;
(2) mencari karakteristik sistem ketika sinyal input dan sinyal output
diketahui; dan (3) mencari sinyal input ketika karakteristik sistem dan
sinyal output diketahui {[}Query{]}.

Hubungan antara input, sistem, dan output dalam analisis sinyal dan
sistem tidak hanya bersifat fungsional, tetapi juga melibatkan prinsip
kausalitas yang mendalam. Sebuah sistem disebut kausal (non-antisipatif)
jika outputnya pada waktu tertentu hanya bergantung pada nilai input
dan/atau output pada waktu saat ini atau sebelumnya, tanpa bergantung
pada nilai input atau output di masa depan.11 Ini berarti bahwa output
tidak dapat mendahului inputnya. Konsekuensi dari prinsip kausalitas ini
sangat signifikan dalam rekayasa sistem nyata. Sistem fisik yang dapat
diimplementasikan secara real-time harus bersifat kausal, karena
mustahil bagi sistem untuk ``mengetahui'' atau ``menggunakan'' informasi
dari masa depan. Misalnya, filter ideal yang dirancang untuk memiliki
respons frekuensi yang sempurna seringkali bersifat non-kausal, yang
berarti tidak dapat direalisasikan secara fisik. Oleh karena itu, dalam
praktik, filter nyata selalu merupakan aproksimasi dari filter ideal
yang bersifat kausal. Pemahaman tentang kausalitas ini menjadi batasan
fundamental yang memandu desain dan analisis sistem di semua domain,
memastikan bahwa model matematis yang dikembangkan sesuai dengan batasan
fisik dan dapat diimplementasikan.

\section{B. Enam Kawasan Analisis Sinyal dan
Sistem}\label{b.-enam-kawasan-analisis-sinyal-dan-sistem}

Untuk memahami karakteristik dan perilaku sinyal dan sistem secara
komprehensif, analisis dilakukan dalam berbagai perspektif atau
``kawasan.'' Terdapat enam kawasan utama yang membentuk fondasi analisis
ini: Waktu Kontinu, Waktu Diskrit, Frekuensi Kontinu, Frekuensi Diskrit,
Kompleks Laplace, dan Kompleks Z {[}Query{]}. Setiap kawasan menawarkan
sudut pandang unik yang menyoroti aspek-aspek berbeda dari sinyal dan
sistem.

Peran sentral dalam menghubungkan dan menavigasi antar kawasan ini
dimainkan oleh berbagai transformasi matematis. Transformasi seperti
Transformasi Fourier, Transformasi Laplace, dan Transformasi Z berfungsi
sebagai jembatan, mengubah representasi sinyal dan sistem dari satu
domain ke domain lain.13 Tujuan utama dari transformasi ini adalah untuk
menyederhanakan analisis dan pemecahan masalah. Misalnya, operasi
konvolusi yang kompleks di domain waktu dapat diubah menjadi perkalian
aljabar sederhana di domain frekuensi atau domain kompleks, yang secara
drastis mengurangi kompleksitas komputasi dan analisis.11 Pemilihan
transformasi yang tepat bergantung pada jenis sinyal (kontinu atau
diskrit) dan tujuan analisis (misalnya, analisis respons transien,
stabilitas, atau konten frekuensi).

\section{II. Peta Pengetahuan Primitif Sinyal dan Sistem (Primitive
Knowledge
Map)}\label{ii.-peta-pengetahuan-primitif-sinyal-dan-sistem-primitive-knowledge-map}

Peta pengetahuan primitif ini menguraikan konsep-konsep dasar sinyal dan
sistem dalam setiap domain, serta karakteristik utamanya.

\section{A. Kawasan Waktu (Time
Domain)}\label{a.-kawasan-waktu-time-domain}

Kawasan waktu adalah representasi paling intuitif dari sinyal,
menggambarkan bagaimana amplitudo sinyal berubah seiring waktu.

\subsection{1. Sinyal Waktu Kontinu (Continuous-Time Signals -
CTS)}\label{sinyal-waktu-kontinu-continuous-time-signals---cts}

Sinyal waktu kontinu, sering disebut sinyal analog, didefinisikan untuk
setiap nilai waktu t dan memiliki nilai real pada keseluruhan rentang
waktu yang ditempatinya, dinotasikan sebagai x(t).3 Variabel
independennya, waktu, bersifat kontinu. Karakteristik umum sinyal waktu
kontinu meliputi apakah sinyal tersebut periodik (berulang setelah
interval waktu

T, yaitu x(t+T)=x(t)) atau aperiodik, apakah memiliki energi terbatas
(sinyal energi) atau daya rata-rata terbatas (sinyal daya), dan apakah
simetris (genap, x(t)=x(−t)) atau antisimetris (ganjil, x(t)=−x(−t)).3
Contoh sinyal dasar dalam domain waktu kontinu adalah fungsi impuls
Dirac (

δ(t)), fungsi step unit (u(t)), fungsi ramp (r(t)), dan sinyal
sinusoidal (Acos(ωt+θ)).3

\subsection{2. Sistem Waktu Kontinu (Continuous-Time Systems -
CTS)}\label{sistem-waktu-kontinu-continuous-time-systems---cts}

Sistem waktu kontinu adalah sistem yang memproses sinyal input waktu
kontinu untuk menghasilkan sinyal output waktu kontinu, dinotasikan
sebagai y(t)=H(x(t)).6 Karakteristik penting dari sistem waktu kontinu
meliputi:

\begin{itemize}
\item
  \textbf{Linearitas:} Sistem linear memenuhi prinsip superposisi, yang
  berarti respons terhadap kombinasi linear input adalah kombinasi
  linear dari respons terhadap setiap input secara terpisah. Ini
  mencakup sifat additivitas (respons terhadap jumlah input adalah
  jumlah respons) dan homogenitas (respons terhadap input yang
  diskalakan adalah respons yang diskalakan).11
\item
  \textbf{Invariansi Waktu (Time-Invariance):} Sistem invarian waktu
  memiliki karakteristik yang tidak berubah seiring waktu. Artinya, jika
  input digeser waktu, outputnya juga akan bergeser waktu dengan jumlah
  yang sama, tanpa perubahan bentuk atau magnitudo.6
\item
  \textbf{Kausalitas:} Output sistem pada waktu tertentu hanya
  bergantung pada input pada waktu saat ini atau masa lalu, tidak pada
  input di masa depan.11
\item
  \textbf{Stabilitas (BIBO - Bounded Input, Bounded Output):} Sistem
  stabil BIBO menjamin bahwa jika input yang diberikan terbatas
  (bounded), maka output yang dihasilkan juga akan terbatas.11
\end{itemize}

\subsection{3. Sinyal Waktu Diskrit (Discrete-Time Signals -
DTS)}\label{sinyal-waktu-diskrit-discrete-time-signals---dts}

Sinyal waktu diskrit didefinisikan hanya pada waktu diskrit, di mana
variabel independennya mengambil nilai integer, dinotasikan sebagai
x{[}n{]}.3 Sinyal-sinyal ini seringkali diperoleh dari proses sampling
sinyal waktu kontinu. Karakteristik umum sinyal waktu diskrit analog
dengan sinyal waktu kontinu, yaitu periodisitas, energi/daya, dan sifat
genap/ganjil.3 Contoh sinyal dasar dalam domain waktu diskrit adalah
sekuen impuls unit (Kronecker delta,

δ{[}n{]}), sekuen step unit (u{[}n{]}), dan sekuen sinusoidal diskrit
(Asin(ω0\hspace{0pt}n+ϕ)).3

\subsection{4. Sistem Waktu Diskrit (Discrete-Time Systems -
DTS)}\label{sistem-waktu-diskrit-discrete-time-systems---dts}

Sistem waktu diskrit adalah sistem yang memproses sinyal input waktu
diskrit untuk menghasilkan sinyal output waktu diskrit, dinotasikan
sebagai y{[}n{]}=H{[}x(n){]}.6 Karakteristik penting dari sistem waktu
diskrit, seperti linearitas, invariansi waktu, kausalitas, dan
stabilitas BIBO, secara konseptual analog dengan sistem waktu kontinu.7

\section{B. Kawasan Frekuensi (Frequency
Domain)}\label{b.-kawasan-frekuensi-frequency-domain}

Kawasan frekuensi memberikan perspektif tentang komponen-komponen
sinusoidal yang membentuk suatu sinyal, mengungkapkan distribusi energi
sinyal di berbagai frekuensi.

\subsection{1. Deret Fourier Waktu Kontinu (Continuous-Time Fourier
Series -
CTFS)}\label{deret-fourier-waktu-kontinu-continuous-time-fourier-series---ctfs}

Deret Fourier Waktu Kontinu (CTFS) digunakan untuk merepresentasikan
sinyal waktu kontinu yang \textbf{periodik} sebagai penjumlahan tak
hingga dari eksponensial kompleks yang berhubungan secara harmonis.39
Ini berarti sinyal periodik dapat diuraikan menjadi komponen-komponen
frekuensi diskrit.

\begin{itemize}
\item
  \textbf{Persamaan Sintesis:}
  x(t)=∑k=−∞∞\hspace{0pt}ak\hspace{0pt}ejkω0\hspace{0pt}t 40

  \begin{itemize}
  \tightlist
  \item
    Di mana ak\hspace{0pt} adalah koefisien Deret Fourier, dan
    ω0\hspace{0pt}=2π/T adalah frekuensi fundamental, dengan T adalah
    periode sinyal.40
  \end{itemize}
\item
  \textbf{Persamaan Analisis (Koefisien Fourier):}
  ak\hspace{0pt}=T1\hspace{0pt}∫T\hspace{0pt}x(t)e−jkω0\hspace{0pt}tdt
  40

  \begin{itemize}
  \tightlist
  \item
    Integral diambil selama satu periode T.
  \end{itemize}
\item
  \textbf{Properti Kunci:} Linearitas, pergeseran waktu (time shifting)
  tidak mengubah koefisien Deret Fourier, dan penskalaan waktu (time
  scaling) mengubah representasi karena perubahan frekuensi fundamental
  tetapi tidak mengubah koefisien.39
\item
  \textbf{Aplikasi:} Analisis spektrum sinyal periodik, seperti
  gelombang persegi atau gelombang segitiga, untuk mengidentifikasi
  komponen frekuensi harmoniknya.43
\end{itemize}

\subsection{2. Transformasi Fourier Waktu Kontinu (Continuous-Time
Fourier Transform -
CTFT)}\label{transformasi-fourier-waktu-kontinu-continuous-time-fourier-transform---ctft}

Transformasi Fourier Waktu Kontinu (CTFT) adalah alat matematis yang
mengubah sinyal waktu kontinu (f(t)) menjadi representasi domain
frekuensi kontinu (X(jω) atau X(Ω)).13 Transformasi ini menguraikan
sinyal menjadi komponen-komponen frekuensi sinusoidalnya, menunjukkan
seberapa besar kontribusi setiap frekuensi terhadap sinyal asli.
Persamaan transformasinya adalah

X(jω)=∫−∞∞\hspace{0pt}x(t)e−jωtdt, dan inversnya adalah
x(t)=2π1\hspace{0pt}∫−∞∞\hspace{0pt}X(jω)ejωtdω.13 Properti kunci CTFT
meliputi linearitas, pergeseran waktu (time shifting), pergeseran
frekuensi (frequency shifting), penskalaan waktu (time scaling), dan
yang terpenting, mengubah operasi konvolusi di domain waktu menjadi
perkalian sederhana di domain frekuensi.13 CTFT banyak diaplikasikan
dalam analisis spektrum sinyal non-periodik, desain filter analog, dan
sistem modulasi.13

\subsection{3. Deret Fourier Waktu Diskrit (Discrete-Time Fourier Series
-
DTFS)}\label{deret-fourier-waktu-diskrit-discrete-time-fourier-series---dtfs}

Deret Fourier Waktu Diskrit (DTFS), juga dikenal sebagai Discrete
Fourier Series (DFS), digunakan untuk merepresentasikan sinyal waktu
diskrit yang \textbf{periodik} dengan periode N.5

\begin{itemize}
\item
  \textbf{Persamaan Sintesis:}
  x{[}n{]}=∑k=0N−1\hspace{0pt}ak\hspace{0pt}ejkω0\hspace{0pt}n 5

  \begin{itemize}
  \tightlist
  \item
    Di mana ω0\hspace{0pt}=2π/N adalah frekuensi fundamental diskrit.5
  \end{itemize}
\item
  \textbf{Persamaan Analisis (Koefisien Fourier):}
  ak\hspace{0pt}=N1\hspace{0pt}∑n=0N−1\hspace{0pt}x{[}n{]}e−jkω0\hspace{0pt}n
  5

  \begin{itemize}
  \tightlist
  \item
    Penjumlahan dilakukan selama satu periode N.
  \end{itemize}
\item
  \textbf{Properti Kunci:} Linearitas, pergeseran sirkular, dan
  dualitas.42
\item
  \textbf{Aplikasi:} Analisis sinyal waktu diskrit periodik, seperti
  sekuens digital yang berulang.
\end{itemize}

\subsection{4. Transformasi Fourier Waktu Diskrit (Discrete-Time Fourier
Transform - DTFT \& Discrete Fourier Transform -
DFT)}\label{transformasi-fourier-waktu-diskrit-discrete-time-fourier-transform---dtft-discrete-fourier-transform---dft}

Untuk sinyal waktu diskrit, terdapat dua transformasi Fourier utama:

\begin{itemize}
\item
  \textbf{Discrete-Time Fourier Transform (DTFT):} DTFT mengubah sekuens
  diskrit (x{[}n{]}) menjadi fungsi frekuensi kontinu yang periodik.21
  Spektrum yang dihasilkan adalah penjumlahan periodik dari CTFT sinyal
  kontinu aslinya yang telah disampel. Persamaan transformasinya adalah

  X(ejω)=∑n=−∞∞\hspace{0pt}x{[}n{]}e−jωn, dan inversnya adalah
  x{[}n{]}=2π1\hspace{0pt}∫−ππ\hspace{0pt}X(ejω)ejωndω.21 DTFT bersifat
  periodik di domain frekuensi.
\item
  \textbf{Discrete Fourier Transform (DFT):} DFT mengubah sekuens
  terbatas dari N sampel yang berjarak sama menjadi sekuens frekuensi
  diskrit dengan panjang yang sama.12 DFT dapat dianggap sebagai sampel
  diskrit dari satu siklus DTFT.21 Persamaan transformasinya adalah

  X{[}k{]}=∑n=0N−1\hspace{0pt}x{[}n{]}e−jN2π\hspace{0pt}kn, dan
  inversnya adalah
  x{[}n{]}=N1\hspace{0pt}∑k=0N−1\hspace{0pt}X{[}k{]}ejN2π\hspace{0pt}kn.20
  Algoritma Fast Fourier Transform (FFT) adalah implementasi efisien
  dari DFT.53 Properti kunci DFT meliputi linearitas dan mengubah
  konvolusi sirkular menjadi perkalian.54 DFT banyak digunakan dalam
  pemrosesan sinyal digital (penyaringan, kompresi data seperti JPEG),
  sistem komunikasi (seperti Orthogonal Frequency Division
  Multiplexing/OFDM), dan analisis spektral.20
\end{itemize}

\section{C. Kawasan Kompleks (Complex
Domain)}\label{c.-kawasan-kompleks-complex-domain}

Kawasan kompleks (s-domain untuk Laplace dan z-domain untuk Z-Transform)
adalah generalisasi dari kawasan frekuensi, memungkinkan analisis sistem
yang lebih luas, termasuk sistem tidak stabil dan penanganan kondisi
awal.

\subsection{1. Transformasi Laplace (Laplace
Transform)}\label{transformasi-laplace-laplace-transform}

Transformasi Laplace mengubah fungsi waktu kontinu (f(t)) menjadi fungsi
variabel kompleks (F(s)) di s-domain atau s-plane.14 Transformasi ini
sangat berguna untuk menganalisis sistem linear dan menyelesaikan
persamaan diferensial. Persamaan transformasi unilateralnya adalah

F(s)=∫0∞\hspace{0pt}f(t)e−stdt.64 Invers Transformasi Laplace
mengembalikan fungsi dari s-domain ke domain waktu.15

\begin{itemize}
\item
  \textbf{Region of Convergence (ROC):} ROC adalah himpunan nilai s di
  mana integral Transformasi Laplace konvergen.68 ROC tidak mengandung
  pole dari

  F(s).68
\item
  \textbf{Analisis Pole-Zero:} Lokasi pole (akar-akar penyebut fungsi
  transfer) dan zero (akar-akar pembilang fungsi transfer) di s-plane
  sangat menentukan dinamika dan stabilitas sistem.69 Sistem dikatakan
  stabil jika semua pole-nya terletak di bidang kiri kompleks (yaitu,
  bagian real dari setiap pole adalah negatif).15
\item
  \textbf{Properti Kunci:} Transformasi Laplace bersifat linear,
  mengubah operasi diferensiasi dan integrasi di domain waktu menjadi
  perkalian dan pembagian sederhana dengan s di domain Laplace, dan
  mengubah konvolusi di domain waktu menjadi perkalian di domain
  Laplace.14 Transformasi ini juga secara alami menangani kondisi awal
  sistem.15
\item
  \textbf{Aplikasi:} Transformasi Laplace banyak diaplikasikan dalam
  analisis dan desain sistem kontrol, analisis rangkaian listrik, dan
  penyelesaian persamaan diferensial linear.14
\end{itemize}

\subsection{2. Transformasi Z
(Z-Transform)}\label{transformasi-z-z-transform}

Transformasi Z adalah analog diskrit dari Transformasi Laplace, yang
mengubah sinyal waktu diskrit (x{[}n{]}) menjadi representasi domain
kompleks (X(z)) di z-domain atau z-plane.73 Persamaan transformasi
bilateralnya adalah

X(z)=∑n=−∞∞\hspace{0pt}x{[}n{]}z−n.73 Invers Transformasi Z
mengembalikan fungsi dari z-domain ke domain waktu diskrit.73

\begin{itemize}
\item
  \textbf{Region of Convergence (ROC):} ROC untuk Transformasi Z adalah
  himpunan nilai z di mana deret Transformasi Z konvergen.78 ROC
  merupakan daerah berbentuk cincin di z-plane dan tidak mengandung
  pole.78
\item
  \textbf{Analisis Pole-Zero:} Lokasi pole dan zero di z-plane
  menentukan stabilitas dan perilaku sistem waktu diskrit.78 Sistem
  dikatakan stabil jika semua pole-nya berada

  \emph{di dalam} lingkaran unit (lingkaran dengan radius satu yang
  berpusat di titik asal) di z-plane.78
\item
  \textbf{Properti Kunci:} Transformasi Z bersifat linear, memiliki
  properti pergeseran waktu, diferensiasi di z-domain, dan mengubah
  konvolusi di domain waktu diskrit menjadi perkalian di domain Z.74
\item
  \textbf{Aplikasi:} Transformasi Z sangat penting dalam analisis dan
  desain sistem kontrol digital, serta dalam pemrosesan sinyal digital
  untuk analisis stabilitas dan desain filter.84
\end{itemize}

\section{D. Konsep Sampling dan
Rekonstruksi}\label{d.-konsep-sampling-dan-rekonstruksi}

Sampling adalah proses fundamental yang menjembatani sinyal waktu
kontinu (analog) dengan sinyal waktu diskrit (digital).

\subsection{Sampling Waktu}\label{sampling-waktu}

Sampling adalah proses mengubah sinyal kontinu menjadi sinyal diskrit
dengan mengambil nilai sinyal pada interval waktu tertentu, T, yang
disebut interval sampling atau periode sampling.28 Frekuensi sampling (

fs\hspace{0pt}=1/T) adalah jumlah sampel per detik.89 Untuk menghindari
distorsi yang dikenal sebagai aliasing, frekuensi sampling harus
memenuhi kriteria yang ditetapkan oleh teorema Nyquist-Shannon.89

\begin{itemize}
\item
  \textbf{Teorema Nyquist-Shannon:} Teorema ini menyatakan bahwa laju
  sampel (fs\hspace{0pt}) harus setidaknya dua kali bandwidth sinyal (B)
  (fs\hspace{0pt}\textgreater2B) untuk memungkinkan rekonstruksi sinyal
  asli yang sempurna dan menghindari aliasing.89 Batas frekuensi

  fs\hspace{0pt}/2 dikenal sebagai frekuensi Nyquist.89
\item
  \textbf{Aliasing:} Aliasing adalah fenomena distorsi yang terjadi
  ketika sinyal kontinu di-sample pada frekuensi yang lebih rendah dari
  laju Nyquist.89
\end{itemize}

Aliasing memiliki implikasi praktis yang mendalam dalam desain sistem
nyata. Fenomena ini muncul karena spektrum sinyal yang telah disampel
merupakan penjumlahan periodik dari spektrum sinyal kontinu aslinya,
berulang pada kelipatan frekuensi sampling.20 Jika frekuensi sampling
terlalu rendah, salinan-salinan spektrum ini akan tumpang tindih,
menyebabkan frekuensi tinggi ``menyamar'' sebagai frekuensi rendah dan
mengakibatkan hilangnya informasi yang tidak dapat dipulihkan.89 Untuk
mengatasi masalah ini, filter anti-aliasing (filter low-pass) harus
diterapkan

\emph{sebelum} proses sampling untuk menghilangkan komponen frekuensi di
atas frekuensi Nyquist.89 Ini adalah langkah krusial dalam konversi
analog-ke-digital (ADC) dan merupakan batasan fundamental dalam
pemrosesan sinyal digital. Pemilihan frekuensi sampling yang tepat,
sesuai dengan teorema Nyquist-Shannon, sangat menentukan kualitas dan
integritas data yang dikonversi. Kegagalan memenuhi kriteria ini akan
menghasilkan distorsi yang signifikan, seperti yang diamati dalam
analisis sinyal audio dengan frekuensi sampling rendah.98

\subsection{Rekonstruksi Sinyal}\label{rekonstruksi-sinyal}

Rekonstruksi sinyal adalah proses kebalikan dari sampling, yaitu
mengembalikan sinyal diskrit menjadi sinyal kontinu.89 Metode yang umum
digunakan untuk rekonstruksi adalah interpolasi sinc atau melalui filter
low-pass ideal.89 Interpolasi sinc secara matematis setara dengan
melewatkan deretan impuls Dirac yang dimodulasi oleh nilai sampel
melalui filter low-pass ideal.89

\subsection{Sampling Frekuensi (Implisit dari Deret Fourier \&
DTFT/DFT)}\label{sampling-frekuensi-implisit-dari-deret-fourier-dtftdft}

Konsep sampling tidak hanya berlaku di domain waktu tetapi juga memiliki
analogi di domain frekuensi, terutama dalam konteks hubungan antara
sinyal kontinu dan diskrit melalui transformasi Fourier.

Ketika sinyal kontinu di-sample di domain waktu, spektrum sinyal diskrit
yang dihasilkan (melalui DTFT) adalah penjumlahan periodik dari spektrum
sinyal kontinu aslinya, dengan periode yang ditentukan oleh frekuensi
sampling.20 Ini menunjukkan bahwa operasi diskritisasi di satu domain
(waktu) menyebabkan periodisitas di domain lainnya (frekuensi).

Peran deret Fourier dalam konteks ini sangat penting. Deret Fourier
digunakan untuk merepresentasikan sinyal periodik (baik waktu kontinu
maupun diskrit) sebagai kombinasi linear dari eksponensial kompleks yang
berhubungan secara harmonis.108 Koefisien deret Fourier ini menunjukkan
komponen frekuensi diskrit yang membentuk sinyal periodik.39

Hubungan antara sampling di domain waktu dan sifat periodik di domain
frekuensi merupakan manifestasi dari dualitas waktu-frekuensi. Dualitas
ini adalah prinsip sentral dalam analisis Fourier yang menyatakan bahwa
operasi di satu domain memiliki efek timbal balik di domain lainnya.
Misalnya:

\begin{itemize}
\item
  Sinyal waktu kontinu periodik (yang dijelaskan oleh Deret Fourier)
  memiliki spektrum frekuensi diskrit.
\item
  Sinyal waktu diskrit aperiodik (yang dijelaskan oleh DTFT) memiliki
  spektrum frekuensi kontinu yang periodik.21
\item
  Sinyal waktu kontinu aperiodik (yang dijelaskan oleh CTFT) memiliki
  spektrum frekuensi kontinu aperiodik.
\item
  Sinyal waktu diskrit periodik (yang dijelaskan oleh DFT/DFS) memiliki
  spektrum frekuensi diskrit periodik.20
\end{itemize}

Pemahaman tentang dualitas ini sangat penting dalam rekayasa sinyal. Ini
menjelaskan mengapa sampling sinyal analog (mengambil sampel pada
titik-titik waktu diskrit) menghasilkan spektrum frekuensi yang berulang
secara periodik.21 Sebaliknya, menganalisis sinyal yang periodik di
domain waktu akan menghasilkan komponen frekuensi yang diskrit.
Pemahaman ini sangat membantu dalam merancang sistem pemrosesan sinyal
digital yang efisien dan akurat, karena secara langsung menginformasikan
hubungan antara operasi di domain waktu dan konsekuensinya di domain
frekuensi.

\section{III. Peta Pemecahan Masalah Sinyal dan Sistem (Problem-Solving
Map)}\label{iii.-peta-pemecahan-masalah-sinyal-dan-sistem-problem-solving-map}

Dalam setiap kawasan analisis (Waktu Kontinu, Waktu Diskrit, Frekuensi
Kontinu, Frekuensi Diskrit, Kompleks Laplace, dan Kompleks Z), terdapat
tiga masalah dasar yang dapat dipecahkan:

\begin{enumerate}
\def\labelenumi{\arabic{enumi}.}
\item
  \textbf{Mencari Output:} Diberikan sinyal input dan karakteristik
  sistem.
\item
  \textbf{Mencari Sistem:} Diberikan sinyal input dan sinyal output.
\item
  \textbf{Mencari Input:} Diberikan karakteristik sistem dan sinyal
  output.
\end{enumerate}

Berikut adalah pendekatan untuk memecahkan masalah-masalah ini di setiap
kawasan.

\section{A. Kawasan Waktu (Time
Domain)}\label{a.-kawasan-waktu-time-domain-1}

\subsection{1. Sinyal dan Sistem Waktu
Kontinu}\label{sinyal-dan-sistem-waktu-kontinu}

\begin{itemize}
\item
  \textbf{Mencari Output (y(t)):}

  \begin{itemize}
  \item
    \textbf{Metode:} Konvolusi Integral. Output y(t) dari sistem Linear
    Time-Invariant (LTI) waktu kontinu adalah konvolusi dari input x(t)
    dan respons impuls sistem h(t).11 Persamaan matematisnya adalah

    y(t)=x(t)∗h(t)=∫−∞∞\hspace{0pt}x(τ)h(t−τ)dτ. Konvolusi merupakan
    operasi fundamental yang menggambarkan bagaimana output sistem LTI
    dihasilkan dari input dan respons impulsnya.
  \end{itemize}
\item
  \textbf{Mencari Sistem (h(t)):}

  \begin{itemize}
  \tightlist
  \item
    \textbf{Metode:} Dekonvolusi atau dengan memberikan input impuls
    unit. Jika input yang diberikan ke sistem adalah impuls Dirac
    (δ(t)), output yang dihasilkan sistem adalah respons impuls h(t).30
    Dekonvolusi adalah proses invers dari konvolusi, yang bertujuan
    untuk menghilangkan efek konvolusi pada data.10
  \end{itemize}
\item
  \textbf{Mencari Input (x(t)):}

  \begin{itemize}
  \tightlist
  \item
    \textbf{Metode:} Dekonvolusi output dengan respons impuls sistem.
    Ini melibatkan operasi invers dari konvolusi.
  \end{itemize}
\end{itemize}

Penyelesaian masalah mencari sistem (respons impuls h(t)) atau mencari
input (x(t)) di domain waktu seringkali melibatkan operasi dekonvolusi.
Dekonvolusi, sebagai proses invers dari konvolusi, bertujuan untuk
memulihkan salah satu sinyal penyusun ketika sinyal hasil konvolusi dan
sinyal lainnya diketahui.125 Meskipun secara matematis didefinisikan,
dekonvolusi di domain waktu memiliki tantangan praktis yang signifikan.
Operasi ini seringkali bersifat

\emph{ill-posed}, artinya solusi mungkin tidak unik, tidak stabil
terhadap noise kecil pada data, atau tidak bergantung secara kontinu
pada data input.38 Dalam aplikasi nyata seperti seismologi atau
pemrosesan citra, noise dan ketidaksempurnaan data dapat memperburuk
masalah ini, membuat dekonvolusi langsung menjadi sangat sulit atau
bahkan tidak mungkin untuk mendapatkan hasil yang akurat.126
Keterbatasan ini mendorong para insinyur untuk memanfaatkan domain
transformasi (seperti domain frekuensi atau kompleks) di mana operasi
konvolusi menjadi perkalian, sehingga operasi inversnya menjadi
pembagian sederhana, meskipun tetap rentan terhadap amplifikasi noise.

\subsection{2. Sinyal dan Sistem Waktu
Diskrit}\label{sinyal-dan-sistem-waktu-diskrit}

\begin{itemize}
\item
  \textbf{Mencari Output (y{[}n{]}):}

  \begin{itemize}
  \tightlist
  \item
    \textbf{Metode:} Konvolusi Sum. Output y{[}n{]} dari sistem LTI
    waktu diskrit adalah konvolusi dari input x{[}n{]} dan respons
    impuls sistem h{[}n{]}.11 Ini adalah analog diskrit dari konvolusi
    integral untuk sinyal waktu kontinu.
  \end{itemize}
\item
  \textbf{Mencari Sistem (h{[}n{]}):}

  \begin{itemize}
  \tightlist
  \item
    \textbf{Metode:} Dekonvolusi diskrit atau dengan memberikan input
    impuls unit (Kronecker delta, δ{[}n{]}). Jika input adalah impuls
    unit, output adalah respons impuls h{[}n{]}.10
  \end{itemize}
\item
  \textbf{Mencari Input (x{[}n{]}):}

  \begin{itemize}
  \tightlist
  \item
    \textbf{Metode:} Dekonvolusi output dengan respons impuls sistem.
  \end{itemize}
\end{itemize}

\section{B. Kawasan Frekuensi (Frequency
Domain)}\label{b.-kawasan-frekuensi-frequency-domain-1}

Analisis di kawasan frekuensi sangat efisien untuk sistem LTI karena
sifat perkalian yang menyederhanakan operasi konvolusi.

\subsection{1. Sinyal dan Sistem Kontinu (menggunakan
CTFT)}\label{sinyal-dan-sistem-kontinu-menggunakan-ctft}

\begin{itemize}
\item
  \textbf{Mencari Output (Y(jω)):}

  \begin{itemize}
  \tightlist
  \item
    \textbf{Metode:} Perkalian di domain frekuensi. Output Y(jω) adalah
    perkalian dari Transformasi Fourier input X(jω) dan respons
    frekuensi sistem H(jω): Y(jω)=X(jω)H(jω).11 Properti ini merupakan
    salah satu alasan utama mengapa analisis di domain frekuensi sangat
    disukai untuk sistem LTI, karena mengubah operasi konvolusi yang
    kompleks di domain waktu menjadi perkalian aljabar sederhana.11
  \end{itemize}
\item
  \textbf{Mencari Sistem (H(jω)):}

  \begin{itemize}
  \item
    \textbf{Metode:} Pembagian di domain frekuensi.
    H(jω)=Y(jω)/X(jω).135 Setelah mendapatkan

    H(jω), invers CTFT dapat dilakukan untuk memperoleh respons impuls
    h(t) di domain waktu.
  \end{itemize}
\item
  \textbf{Mencari Input (X(jω)):}

  \begin{itemize}
  \item
    \textbf{Metode:} Pembagian di domain frekuensi. X(jω)=Y(jω)/H(jω).3
    Setelah mendapatkan

    X(jω), invers CTFT dapat dilakukan untuk memperoleh input x(t) di
    domain waktu.
  \end{itemize}
\end{itemize}

Efisiensi analisis sistem LTI di domain frekuensi berasal dari sifat
fundamental Transformasi Fourier yang mengubah operasi konvolusi di
domain waktu menjadi perkalian di domain frekuensi.11 Konvolusi, yang
secara matematis melibatkan integral atau penjumlahan yang rumit,
menjadi operasi aljabar yang jauh lebih sederhana (

Y(jω)=X(jω)H(jω)).13 Penyederhanaan ini tidak hanya mengurangi beban
komputasi tetapi juga memberikan pemahaman konseptual yang lebih jelas
tentang bagaimana sistem memodifikasi komponen frekuensi sinyal input.
Misalnya, respons frekuensi sistem

H(jω) secara langsung menunjukkan bagaimana sistem melemahkan atau
menguatkan frekuensi tertentu.133 Kemampuan untuk dengan cepat
menentukan output, sistem, atau input melalui perkalian atau pembagian
di domain frekuensi merupakan perubahan paradigma yang signifikan dalam
rekayasa, memungkinkan desain, analisis, dan optimasi filter, sistem
komunikasi, dan sistem kontrol yang jauh lebih cepat dan intuitif.

\subsection{2. Sinyal dan Sistem Diskrit (menggunakan
DTFT/DFT)}\label{sinyal-dan-sistem-diskrit-menggunakan-dtftdft}

\begin{itemize}
\item
  \textbf{Mencari Output (Y(ejω) atau Y{[}k{]}):}

  \begin{itemize}
  \tightlist
  \item
    \textbf{Metode:} Perkalian di domain frekuensi. Output
    Y(ejω)=X(ejω)H(ejω) untuk DTFT, atau Y{[}k{]}=X{[}k{]}H{[}k{]} untuk
    DFT.12 Ini adalah analog diskrit dari perkalian di domain frekuensi
    kontinu.
  \end{itemize}
\item
  \textbf{Mencari Sistem (H(ejω) atau H{[}k{]}):}

  \begin{itemize}
  \tightlist
  \item
    \textbf{Metode:} Pembagian di domain frekuensi. H(ejω)=Y(ejω)/X(ejω)
    atau H{[}k{]}=Y{[}k{]}/X{[}k{]}.113
  \end{itemize}
\item
  \textbf{Mencari Input (X(ejω) atau X{[}k{]}):}

  \begin{itemize}
  \tightlist
  \item
    \textbf{Metode:} Pembagian di domain frekuensi. X(ejω)=Y(ejω)/H(ejω)
    atau X{[}k{]}=Y{[}k{]}/H{[}k{]}.114
  \end{itemize}
\end{itemize}

\section{C. Kawasan Kompleks (Complex
Domain)}\label{c.-kawasan-kompleks-complex-domain-1}

Kawasan kompleks (Laplace dan Z-Transform) menyediakan kerangka analisis
yang lebih umum dan kuat, terutama untuk sistem yang tidak stabil atau
yang memerlukan pertimbangan kondisi awal.

\subsection{1. Sinyal dan Sistem Kontinu (menggunakan Transformasi
Laplace)}\label{sinyal-dan-sistem-kontinu-menggunakan-transformasi-laplace}

\begin{itemize}
\item
  \textbf{Mencari Output (Y(s)):}

  \begin{itemize}
  \item
    \textbf{Metode:} Perkalian di domain Laplace. Y(s)=X(s)H(s).15
    Setelah mendapatkan

    Y(s), invers Transformasi Laplace dilakukan untuk mendapatkan y(t)
    di domain waktu. Transformasi Laplace sangat efektif karena mengubah
    persamaan diferensial linear menjadi persamaan aljabar, yang jauh
    lebih mudah dipecahkan.62
  \end{itemize}
\item
  \textbf{Mencari Sistem (H(s)):}

  \begin{itemize}
  \item
    \textbf{Metode:} Pembagian di domain Laplace. H(s)=Y(s)/X(s).9
    Setelah mendapatkan

    H(s), invers Transformasi Laplace dapat dilakukan untuk memperoleh
    respons impuls h(t) di domain waktu.
  \end{itemize}
\item
  \textbf{Mencari Input (X(s)):}

  \begin{itemize}
  \tightlist
  \item
    \textbf{Metode:} Pembagian di domain Laplace. X(s)=Y(s)/H(s).
    Setelah mendapatkan X(s), invers Transformasi Laplace dapat
    dilakukan untuk memperoleh input x(t) di domain waktu.
  \end{itemize}
\end{itemize}

\subsection{2. Sinyal dan Sistem Diskrit (menggunakan Transformasi
Z)}\label{sinyal-dan-sistem-diskrit-menggunakan-transformasi-z}

\begin{itemize}
\item
  \textbf{Mencari Output (Y(z)):}

  \begin{itemize}
  \item
    \textbf{Metode:} Perkalian di domain Z. Y(z)=X(z)H(z).84 Setelah
    mendapatkan

    Y(z), invers Transformasi Z dilakukan untuk mendapatkan y{[}n{]} di
    domain waktu diskrit. Transformasi Z mengubah persamaan beda menjadi
    persamaan aljabar, menyederhanakan penyelesaian.84
  \end{itemize}
\item
  \textbf{Mencari Sistem (H(z)):}

  \begin{itemize}
  \item
    \textbf{Metode:} Pembagian di domain Z. H(z)=Y(z)/X(z).19 Setelah
    mendapatkan

    H(z), invers Transformasi Z dapat dilakukan untuk memperoleh respons
    impuls h{[}n{]} di domain waktu diskrit.
  \end{itemize}
\item
  \textbf{Mencari Input (X(z)):}

  \begin{itemize}
  \item
    \textbf{Metode:} Pembagian di domain Z. X(z)=Y(z)/H(z).73 Setelah
    mendapatkan

    X(z), invers Transformasi Z dapat dilakukan untuk memperoleh input
    x{[}n{]} di domain waktu diskrit.
  \end{itemize}
\end{itemize}

\section{D. Transformasi Antar Kawasan (Inter-Domain
Transformations)}\label{d.-transformasi-antar-kawasan-inter-domain-transformations}

Transformasi antar kawasan adalah alat vital yang memungkinkan analisis
sinyal dan sistem dari berbagai perspektif, memilih domain yang paling
sesuai untuk masalah tertentu.

\subsection{1. Waktu Kontinu ↔ Frekuensi
Kontinu}\label{waktu-kontinu-frekuensi-kontinu}

Hubungan ini diatur oleh Transformasi Fourier Kontinu (CTFT) dan Invers
CTFT.13 CTFT mengubah sinyal waktu kontinu menjadi spektrum frekuensi
kontinu, dan Invers CTFT melakukan sebaliknya.

\subsection{2. Waktu Diskrit ↔ Frekuensi
Diskrit}\label{waktu-diskrit-frekuensi-diskrit}

Hubungan ini diatur oleh Transformasi Fourier Waktu Diskrit (DTFT) dan
Discrete Fourier Transform (DFT), beserta inversnya.5 DTFT menghasilkan
spektrum frekuensi kontinu yang periodik dari sinyal waktu diskrit,
sementara DFT menyediakan sampel diskrit dari spektrum ini.

\subsection{3. Waktu Kontinu ↔ Kompleks
Laplace}\label{waktu-kontinu-kompleks-laplace}

Transformasi Laplace dan Invers Laplace menjembatani domain waktu
kontinu dengan domain kompleks s.14

Generalisasi analisis sistem LTI dengan domain kompleks, melalui
Transformasi Laplace dan Transformasi Z, memberikan kerangka kerja yang
lebih kuat dibandingkan hanya menggunakan Transformasi Fourier.
Transformasi Fourier, meskipun sangat berguna untuk analisis konten
frekuensi, terbatas pada sinyal dan sistem yang stabil (yaitu, integral
Fourier-nya konvergen) dan hanya beroperasi pada sumbu imajiner
(jω-axis) di s-plane.14 Transformasi Laplace memperluas analisis ini ke
seluruh bidang kompleks (

s-plane), memungkinkan karakterisasi sistem yang tidak stabil (pole di
bidang kanan kompleks) dan analisis respons transien serta kondisi awal
sistem.14 ROC dari Transformasi Laplace secara langsung menunjukkan
kausalitas dan stabilitas sistem; untuk sistem kausal, ROC terletak di
sebelah kanan pole paling kanan.147 Perluasan ini sangat penting dalam
desain sistem kontrol, di mana stabilitas adalah pertimbangan utama.
Transformasi Laplace mengubah persamaan diferensial menjadi persamaan
aljabar, menyederhanakan proses penyelesaian secara signifikan.62

\subsection{4. Waktu Diskrit ↔ Kompleks
Z}\label{waktu-diskrit-kompleks-z}

Transformasi Z dan Invers Z menghubungkan domain waktu diskrit dengan
domain kompleks z.78 Transformasi Z adalah analog diskrit dari
Transformasi Laplace, membawa manfaat yang serupa untuk analisis sistem
waktu diskrit.

\subsection{5. Frekuensi Kontinu ↔ Kompleks
Laplace}\label{frekuensi-kontinu-kompleks-laplace}

Sumbu jω di s-plane Transformasi Laplace secara langsung berhubungan
dengan domain frekuensi kontinu dari Transformasi Fourier.14 Ini berarti
bahwa Transformasi Fourier dapat dianggap sebagai kasus khusus dari
Transformasi Laplace ketika bagian real dari variabel kompleks

s adalah nol.

\subsection{6. Frekuensi Diskrit ↔ Kompleks
Z}\label{frekuensi-diskrit-kompleks-z}

Lingkaran unit di z-plane Transformasi Z berhubungan dengan domain
frekuensi diskrit dari DTFT.78 Jika ROC dari Transformasi Z mencakup
lingkaran unit, maka DTFT sinyal tersebut ada.

\section{IV. Visualisasi Peta Pengetahuan (Knowledge Map
Visualization)}\label{iv.-visualisasi-peta-pengetahuan-knowledge-map-visualization}

Untuk mengorganisir dan menyajikan informasi yang kompleks ini secara
efektif, visualisasi dalam bentuk peta pengetahuan sangatlah bermanfaat.
Dua jenis peta dapat disusun: Peta Pengetahuan Primitif (konseptual) dan
Peta Pemecahan Masalah (aplikatif).

\section{A. Primitive Knowledge Map
(Konseptual)}\label{a.-primitive-knowledge-map-konseptual}

Peta pengetahuan primitif berfungsi sebagai gambaran konseptual
hierarkis dari sinyal dan sistem.

\begin{itemize}
\item
  \textbf{Struktur:} Peta ini memiliki struktur hierarkis, dimulai dari
  konsep inti ``Sinyal \& Sistem'' sebagai node pusat.
\item
  \textbf{Nodes:} Level berikutnya akan bercabang menjadi ``Kawasan
  Waktu,'' ``Kawasan Frekuensi,'' dan ``Kawasan Kompleks.'' Di bawah
  setiap kawasan, akan ada node untuk ``Sinyal Kontinu,'' ``Sistem
  Kontinu,'' ``Sinyal Diskrit,'' dan ``Sistem Diskrit'' (untuk Kawasan
  Waktu); ``Deret Fourier Waktu Kontinu,'' ``CTFT,'' ``Deret Fourier
  Waktu Diskrit,'' dan ``DTFT/DFT'' (untuk Kawasan Frekuensi); serta
  ``Transformasi Laplace'' dan ``Transformasi Z'' (untuk Kawasan
  Kompleks). Node yang lebih rendah akan merinci properti sinyal
  (periodik, energi/daya, genap/ganjil), properti sistem (linearitas,
  invariansi waktu, kausalitas, stabilitas), definisi Region of
  Convergence (ROC), dan analisis pole-zero.
\item
  \textbf{Links:} Hubungan antar node akan ditunjukkan dengan panah
  berlabel. Misalnya, ``terdiri dari'' atau ``dapat diklasifikasikan
  sebagai'' akan menghubungkan level hierarkis. ``Dihubungkan oleh''
  atau ``menggunakan'' akan menghubungkan konsep sampling dan
  transformasi. ``Memiliki properti'' akan menghubungkan sinyal/sistem
  dengan karakteristiknya.
\item
  \textbf{Visual Elements:} Penggunaan warna yang berbeda untuk setiap
  domain utama (Waktu, Frekuensi, Kompleks) dan sub-domain (Kontinu,
  Diskrit) akan meningkatkan kejelasan visual. Simbol standar seperti
  kotak atau lingkaran untuk node dan panah berlabel untuk hubungan akan
  digunakan.149
\item
  \textbf{Suggested Tool:} Alat seperti Miro, Lucidchart, atau
  MindMeister sangat cocok untuk membuat peta kolaboratif dan menarik
  secara visual.154 MathWhiteboard dapat digunakan untuk
  mengintegrasikan ekspresi matematis langsung ke dalam peta.155
\end{itemize}

\section{B. Problem-Solving Map
(Aplikasi)}\label{b.-problem-solving-map-aplikasi}

Peta pemecahan masalah akan memvisualisasikan pendekatan sistematis
untuk menyelesaikan tiga masalah dasar dalam setiap kawasan.

\begin{itemize}
\item
  \textbf{Struktur:} Peta ini dapat berupa matriks atau flowchart,
  dengan domain sebagai baris/kolom dan masalah dasar sebagai entri.
\item
  \textbf{Nodes:} Node akan mencakup ``Mencari Output,'' ``Mencari
  Sistem,'' dan ``Mencari Input'' sebagai masalah utama. Di bawahnya,
  akan ada node untuk setiap dari enam kawasan, dan di bawahnya lagi,
  node detail untuk metode spesifik (misalnya, Konvolusi Integral/Sum,
  Dekonvolusi, Transformasi/Invers Transformasi, Persamaan
  Diferensial/Beda, Fungsi Transfer, Analisis Pole-Zero).
\item
  \textbf{Links:} Panah akan menunjukkan aliran proses atau
  ketergantungan antar metode.
\item
  \textbf{Visual Elements:} Simbol flowchart standar (mulai/akhir,
  proses, keputusan, data, panah) akan digunakan untuk memandu alur
  pemikiran dalam memecahkan masalah.156
\item
  \textbf{Transformasi Antar Kawasan:} Transformasi ini akan
  direpresentasikan sebagai jembatan atau panah besar yang menghubungkan
  domain-domain yang berbeda, dengan label transformasinya (CTFT,
  DTFT/DFT, Laplace, Z-Transform, Invers Transformasi), menunjukkan
  bagaimana berpindah antar domain untuk menyederhanakan masalah.
\end{itemize}

\section{V. Kesimpulan dan
Rekomendasi}\label{v.-kesimpulan-dan-rekomendasi}

\section{A. Ringkasan Integrasi
Konsep}\label{a.-ringkasan-integrasi-konsep}

Konsep dasar sinyal dan sistem adalah landasan bagi banyak inovasi
teknologi modern. Laporan ini telah menguraikan bagaimana sinyal dan
sistem berinteraksi, membentuk hubungan input-sistem-output yang
fundamental. Pemahaman tentang enam kawasan analisis (waktu kontinu,
waktu diskrit, frekuensi kontinu, frekuensi diskrit, kompleks Laplace,
dan kompleks Z) sangat penting untuk analisis yang komprehensif.
Transformasi matematis berfungsi sebagai alat yang tak ternilai untuk
berpindah antar kawasan ini, seringkali mengubah masalah yang kompleks
(seperti konvolusi di domain waktu) menjadi operasi aljabar yang lebih
sederhana (seperti perkalian di domain frekuensi atau kompleks). Konsep
sampling dan rekonstruksi, yang diatur oleh teorema Nyquist-Shannon,
merupakan jembatan krusial antara dunia analog dan digital, dengan
implikasi signifikan terhadap integritas sin

\bookmarksetup{startatroot}

\chapter{Knowledge}\label{knowledge}

Tentu, saya akan mengidentifikasi pernyataan ABCD (Actor Behaves under a
Condition to a certain Degree) dari laporan dan menyusunnya dalam daftar
dengan label mnemonik untuk kemudahan mengingat dan pengindeksan.

Berikut adalah daftar pernyataan ABCD yang ditemukan dalam laporan:

\begin{enumerate}
\def\labelenumi{\arabic{enumi}.}
\item
  \textbf{SISTEM-PROSES-INPUT-OUTPUT}

  \begin{itemize}
  \item
    \textbf{A}ktor: Sistem
  \item
    \textbf{B}erperilaku: memproses
  \item
    \textbf{C}ondisi: sinyal input
  \item
    \textbf{D}erajat: menghasilkan sinyal output
  \item
    \textbf{Pernyataan Lengkap:} Sistem memproses sinyal input untuk
    menghasilkan sinyal output.
  \end{itemize}
\item
  \textbf{CARI-OUTPUT-INPUT-SISTEM-DIKETAHUI}

  \begin{itemize}
  \item
    \textbf{A}ktor: Analis/Insinyur (tersirat)
  \item
    \textbf{B}erperilaku: mencari sinyal output
  \item
    \textbf{C}ondisi: sinyal input dan karakteristik sistem diketahui
  \item
    \textbf{D}erajat: (dengan sukses)
  \item
    \textbf{Pernyataan Lengkap:} (Analis/Insinyur) mencari sinyal output
    ketika sinyal input dan karakteristik sistem diketahui.
  \end{itemize}
\item
  \textbf{CARI-SISTEM-INPUT-OUTPUT-DIKETAHUI}

  \begin{itemize}
  \item
    \textbf{A}ktor: Analis/Insinyur (tersirat)
  \item
    \textbf{B}erperilaku: mencari karakteristik sistem
  \item
    \textbf{C}ondisi: sinyal input dan sinyal output diketahui
  \item
    \textbf{D}erajat: (dengan sukses)
  \item
    \textbf{Pernyataan Lengkap:} (Analis/Insinyur) mencari karakteristik
    sistem ketika sinyal input dan sinyal output diketahui.
  \end{itemize}
\item
  \textbf{CARI-INPUT-SISTEM-OUTPUT-DIKETAHUI}

  \begin{itemize}
  \item
    \textbf{A}ktor: Analis/Insinyur (tersirat)
  \item
    \textbf{B}erperilaku: mencari sinyal input
  \item
    \textbf{C}ondisi: karakteristik sistem dan sinyal output diketahui
  \item
    \textbf{D}erajat: (dengan sukses)
  \item
    \textbf{Pernyataan Lengkap:} (Analis/Insinyur) mencari sinyal input
    ketika karakteristik sistem dan sinyal output diketahui.
  \end{itemize}
\item
  \textbf{SISTEM-KAUSAL-OUTPUT-MASA-LALU}

  \begin{itemize}
  \item
    \textbf{A}ktor: Sebuah sistem
  \item
    \textbf{B}erperilaku: disebut kausal
  \item
    \textbf{C}ondisi: jika outputnya pada waktu tertentu hanya
    bergantung pada nilai input dan/atau output pada waktu saat ini atau
    sebelumnya, tanpa bergantung pada nilai input atau output di masa
    depan
  \item
    \textbf{D}erajat: (sepenuhnya)
  \item
    \textbf{Pernyataan Lengkap:} Sebuah sistem disebut kausal
    (non-antisipatif) jika outputnya pada waktu tertentu hanya
    bergantung pada nilai input dan/atau output pada waktu saat ini atau
    sebelumnya, tanpa bergantung pada nilai input atau output di masa
    depan.
  \end{itemize}
\item
  \textbf{SISTEM-FISIK-REALTIME-KAUSAL}

  \begin{itemize}
  \item
    \textbf{A}ktor: Sistem fisik
  \item
    \textbf{B}erperilaku: harus bersifat kausal
  \item
    \textbf{C}ondisi: (agar dapat) diimplementasikan secara real-time
  \item
    \textbf{D}erajat: (sepenuhnya)
  \item
    \textbf{Pernyataan Lengkap:} Sistem fisik yang dapat
    diimplementasikan secara real-time harus bersifat kausal.
  \end{itemize}
\item
  \textbf{TRANSFORMASI-JEMBATAN-DOMAIN-PERUBAHAN}

  \begin{itemize}
  \item
    \textbf{A}ktor: Transformasi (Fourier, Laplace, Z)
  \item
    \textbf{B}erperilaku: berfungsi sebagai jembatan
  \item
    \textbf{C}ondisi: (ketika diterapkan)
  \item
    \textbf{D}erajat: mengubah representasi sinyal dan sistem dari satu
    domain ke domain lain
  \item
    \textbf{Pernyataan Lengkap:} Transformasi seperti Transformasi
    Fourier, Transformasi Laplace, dan Transformasi Z berfungsi sebagai
    jembatan, mengubah representasi sinyal dan sistem dari satu domain
    ke domain lain.
  \end{itemize}
\item
  \textbf{TRANSFORMASI-TUJUAN-PENYEDERHANAAN}

  \begin{itemize}
  \item
    \textbf{A}ktor: Transformasi ini
  \item
    \textbf{B}erperilaku: bertujuan
  \item
    \textbf{C}ondisi: (ketika digunakan)
  \item
    \textbf{D}erajat: menyederhanakan analisis dan pemecahan masalah
  \item
    \textbf{Pernyataan Lengkap:} Tujuan utama dari transformasi ini
    adalah untuk menyederhanakan analisis dan pemecahan masalah.
  \end{itemize}
\item
  \textbf{SINYAL-CT-PERIODIK-DEFINISI}

  \begin{itemize}
  \item
    \textbf{A}ktor: Suatu sinyal waktu kontinu x(t)
  \item
    \textbf{B}erperilaku: dikatakan periodik
  \item
    \textbf{C}ondisi: jika x(t + T) = x(t) untuk semua nilai t, dari −∞
    \textless{} t \textless{} ∞
  \item
    \textbf{D}erajat: (sepenuhnya)
  \item
    \textbf{Pernyataan Lengkap:} Suatu sinyal waktu kontinu x(t)
    dikatakan periodik terhadap waktu dengan periode T jika x(t + T) =
    x(t) untuk semua nilai t, dari −∞ \textless{} t \textless{} ∞.
  \end{itemize}
\item
  \textbf{SISTEM-LINEAR-SUPERPOSISI}

  \begin{itemize}
  \item
    \textbf{A}ktor: Sistem linear
  \item
    \textbf{B}erperilaku: memenuhi
  \item
    \textbf{C}ondisi: (jika) respons terhadap kombinasi linear input
    adalah kombinasi linear dari respons terhadap setiap input secara
    terpisah
  \item
    \textbf{D}erajat: prinsip superposisi
  \item
    \textbf{Pernyataan Lengkap:} Sistem linear memenuhi prinsip
    superposisi.
  \end{itemize}
\item
  \textbf{SISTEM-INVARIAN-WAKTU-KARAKTERISTIK}

  \begin{itemize}
  \item
    \textbf{A}ktor: Sistem invarian waktu
  \item
    \textbf{B}erperilaku: memiliki karakteristik
  \item
    \textbf{C}ondisi: (jika) tidak berubah seiring waktu
  \item
    \textbf{D}erajat: (sepenuhnya)
  \item
    \textbf{Pernyataan Lengkap:} Sistem invarian waktu memiliki
    karakteristik yang tidak berubah seiring waktu.
  \end{itemize}
\item
  \textbf{SISTEM-BIBO-STABIL-OUTPUT-TERBATAS}

  \begin{itemize}
  \item
    \textbf{A}ktor: Sistem stabil BIBO
  \item
    \textbf{B}erperilaku: menjamin
  \item
    \textbf{C}ondisi: jika input yang diberikan terbatas (bounded)
  \item
    \textbf{D}erajat: output yang dihasilkan juga akan terbatas
  \item
    \textbf{Pernyataan Lengkap:} Sistem stabil BIBO menjamin bahwa jika
    input yang diberikan terbatas (bounded), maka output yang dihasilkan
    juga akan terbatas.
  \end{itemize}
\item
  \textbf{CTFS-REPRESENTASI-SINYAL-PERIODIK}

  \begin{itemize}
  \item
    \textbf{A}ktor: Deret Fourier Waktu Kontinu (CTFS)
  \item
    \textbf{B}erperilaku: digunakan
  \item
    \textbf{C}ondisi: untuk sinyal waktu kontinu yang periodik
  \item
    \textbf{D}erajat: sebagai penjumlahan tak hingga dari eksponensial
    kompleks yang berhubungan secara harmonis
  \item
    \textbf{Pernyataan Lengkap:} Deret Fourier Waktu Kontinu (CTFS)
    digunakan untuk merepresentasikan sinyal waktu kontinu yang
    \textbf{periodik} sebagai penjumlahan tak hingga dari eksponensial
    kompleks yang berhubungan secara harmonis.
  \end{itemize}
\item
  \textbf{CTFT-UBAH-WAKTU-FREKUENSI}

  \begin{itemize}
  \item
    \textbf{A}ktor: Transformasi Fourier Waktu Kontinu (CTFT)
  \item
    \textbf{B}erperilaku: mengubah
  \item
    \textbf{C}ondisi: sinyal waktu kontinu (f(t))
  \item
    \textbf{D}erajat: menjadi representasi domain frekuensi kontinu
    (X(jω) atau X(Ω))
  \item
    \textbf{Pernyataan Lengkap:} Transformasi Fourier Waktu Kontinu
    (CTFT) adalah alat matematis yang mengubah sinyal waktu kontinu
    (f(t)) menjadi representasi domain frekuensi kontinu (X(jω) atau
    X(Ω)).
  \end{itemize}
\item
  \textbf{CTFT-KONVOLUSI-PERKALIAN}

  \begin{itemize}
  \item
    \textbf{A}ktor: CTFT
  \item
    \textbf{B}erperilaku: mengubah
  \item
    \textbf{C}ondisi: operasi konvolusi di domain waktu
  \item
    \textbf{D}erajat: menjadi perkalian sederhana di domain frekuensi
  \item
    \textbf{Pernyataan Lengkap:} Properti kunci CTFT mengubah operasi
    konvolusi di domain waktu menjadi perkalian sederhana di domain
    frekuensi.
  \end{itemize}
\item
  \textbf{LAPLACE-UBAH-WAKTU-S-DOMAIN}

  \begin{itemize}
  \item
    \textbf{A}ktor: Transformasi Laplace
  \item
    \textbf{B}erperilaku: mengubah
  \item
    \textbf{C}ondisi: fungsi waktu kontinu (f(t))
  \item
    \textbf{D}erajat: menjadi fungsi variabel kompleks (F(s)) di
    s-domain
  \item
    \textbf{Pernyataan Lengkap:} Transformasi Laplace mengubah fungsi
    waktu kontinu (f(t)) menjadi fungsi variabel kompleks (F(s)) di
    s-domain atau s-plane.
  \end{itemize}
\item
  \textbf{SISTEM-STABIL-POLE-KIRI}

  \begin{itemize}
  \item
    \textbf{A}ktor: Sistem
  \item
    \textbf{B}erperilaku: dikatakan stabil
  \item
    \textbf{C}ondisi: jika semua pole-nya terletak di bidang kiri
    kompleks (yaitu, bagian real dari setiap pole adalah negatif)
  \item
    \textbf{D}erajat: (sepenuhnya)
  \item
    \textbf{Pernyataan Lengkap:} Sistem dikatakan stabil jika semua
    pole-nya terletak di bidang kiri kompleks.
  \end{itemize}
\item
  \textbf{Z-TRANSFORM-UBAH-WAKTU-Z-DOMAIN}

  \begin{itemize}
  \item
    \textbf{A}ktor: Transformasi Z
  \item
    \textbf{B}erperilaku: mengubah
  \item
    \textbf{C}ondisi: sinyal waktu diskrit (x{[}n{]})
  \item
    \textbf{D}erajat: menjadi representasi domain kompleks (X(z)) di
    z-domain
  \item
    \textbf{Pernyataan Lengkap:} Transformasi Z mengubah sinyal waktu
    diskrit (x{[}n{]}) menjadi representasi domain kompleks (X(z)) di
    z-domain atau z-plane.
  \end{itemize}
\item
  \textbf{SISTEM-STABIL-POLE-UNIT-LINGKARAN}

  \begin{itemize}
  \item
    \textbf{A}ktor: Sistem
  \item
    \textbf{B}erperilaku: dikatakan stabil
  \item
    \textbf{C}ondisi: jika semua pole-nya berada \emph{di dalam}
    lingkaran unit (lingkaran dengan radius satu yang berpusat di titik
    asal) di z-plane
  \item
    \textbf{D}erajat: (sepenuhnya)
  \item
    \textbf{Pernyataan Lengkap:} Sistem dikatakan stabil jika semua
    pole-nya berada \emph{di dalam} lingkaran unit (lingkaran dengan
    radius satu yang berpusat di titik asal) di z-plane.
  \end{itemize}
\item
  \textbf{SAMPLING-HINDARI-ALIASING-NYQUIST}

  \begin{itemize}
  \item
    \textbf{A}ktor: Frekuensi sampling
  \item
    \textbf{B}erperilaku: harus memenuhi
  \item
    \textbf{C}ondisi: untuk menghindari distorsi yang dikenal sebagai
    aliasing
  \item
    \textbf{D}erajat: kriteria yang ditetapkan oleh teorema
    Nyquist-Shannon
  \item
    \textbf{Pernyataan Lengkap:} Untuk menghindari distorsi yang dikenal
    sebagai aliasing, frekuensi sampling harus memenuhi kriteria yang
    ditetapkan oleh teorema Nyquist-Shannon.
  \end{itemize}
\item
  \textbf{ALIASING-FREKUENSI-RENDAH-DISTORSI}

  \begin{itemize}
  \item
    \textbf{A}ktor: Salinan-salinan spektrum
  \item
    \textbf{B}erperilaku: akan tumpang tindih
  \item
    \textbf{C}ondisi: jika frekuensi sampling terlalu rendah
  \item
    \textbf{D}erajat: menyebabkan frekuensi tinggi ``menyamar'' sebagai
    frekuensi rendah dan mengakibatkan hilangnya informasi yang tidak
    dapat dipulihkan
  \item
    \textbf{Pernyataan Lengkap:} Jika frekuensi sampling terlalu rendah,
    salinan-salinan spektrum ini akan tumpang tindih, menyebabkan
    frekuensi tinggi ``menyamar'' sebagai frekuensi rendah dan
    mengakibatkan hilangnya informasi yang tidak dapat dipulihkan.
  \end{itemize}
\item
  \textbf{OUTPUT-CT-LTI-KONVOLUSI}

  \begin{itemize}
  \item
    \textbf{A}ktor: Output y(t)
  \item
    \textbf{B}erperilaku: adalah
  \item
    \textbf{C}ondisi: dari sistem Linear Time-Invariant (LTI) waktu
    kontinu dengan input x(t) dan respons impuls sistem h(t)
  \item
    \textbf{D}erajat: konvolusi
  \item
    \textbf{Pernyataan Lengkap:} Output y(t) dari sistem Linear
    Time-Invariant (LTI) waktu kontinu adalah konvolusi dari input x(t)
    dan respons impuls sistem h(t).
  \end{itemize}
\item
  \textbf{SISTEM-RESPONS-IMPULS-DIRAC}

  \begin{itemize}
  \item
    \textbf{A}ktor: Output yang dihasilkan sistem
  \item
    \textbf{B}erperilaku: adalah
  \item
    \textbf{C}ondisi: jika input yang diberikan ke sistem adalah impuls
    Dirac (δ(t))
  \item
    \textbf{D}erajat: respons impuls h(t)
  \item
    \textbf{Pernyataan Lengkap:} Jika input yang diberikan ke sistem
    adalah impuls Dirac (δ(t)), output yang dihasilkan sistem adalah
    respons impuls h(t).
  \end{itemize}
\item
  \textbf{OUTPUT-DT-LTI-KONVOLUSI}

  \begin{itemize}
  \item
    \textbf{A}ktor: Output y{[}n{]}
  \item
    \textbf{B}erperilaku: adalah
  \item
    \textbf{C}ondisi: dari sistem LTI waktu diskrit dengan input
    x{[}n{]} dan respons impuls sistem h{[}n{]}
  \item
    \textbf{D}erajat: konvolusi
  \item
    \textbf{Pernyataan Lengkap:} Output y{[}n{]} dari sistem LTI waktu
    diskrit adalah konvolusi dari input x{[}n{]} dan respons impuls
    sistem h{[}n{]}.
  \end{itemize}
\item
  \textbf{OUTPUT-FREKUENSI-CTFT-PERKALIAN}

  \begin{itemize}
  \item
    \textbf{A}ktor: Output Y(jω)
  \item
    \textbf{B}erperilaku: adalah
  \item
    \textbf{C}ondisi: (ketika menggunakan CTFT)
  \item
    \textbf{D}erajat: perkalian dari Transformasi Fourier input X(jω)
    dan respons frekuensi sistem H(jω)
  \item
    \textbf{Pernyataan Lengkap:} Output Y(jω) adalah perkalian dari
    Transformasi Fourier input X(jω) dan respons frekuensi sistem H(jω).
  \end{itemize}
\item
  \textbf{ANALISIS-LTI-EFISIEN-FREKUENSI}

  \begin{itemize}
  \item
    \textbf{A}ktor: Analisis sistem LTI
  \item
    \textbf{B}erperilaku: menjadi efisien
  \item
    \textbf{C}ondisi: di domain frekuensi
  \item
    \textbf{D}erajat: karena sifat fundamental Transformasi Fourier yang
    mengubah operasi konvolusi di domain waktu menjadi perkalian di
    domain frekuensi
  \item
    \textbf{Pernyataan Lengkap:} Efisiensi analisis sistem LTI di domain
    frekuensi berasal dari sifat fundamental Transformasi Fourier yang
    mengubah operasi konvolusi di domain waktu menjadi perkalian di
    domain frekuensi.
  \end{itemize}
\item
  \textbf{LAPLACE-EFEKTIF-PD-ALJABAR}

  \begin{itemize}
  \item
    \textbf{A}ktor: Transformasi Laplace
  \item
    \textbf{B}erperilaku: menjadi sangat efektif
  \item
    \textbf{C}ondisi: (ketika diterapkan pada) persamaan diferensial
    linear
  \item
    \textbf{D}erajat: karena mengubahnya menjadi persamaan aljabar yang
    jauh lebih mudah dipecahkan
  \item
    \textbf{Pernyataan Lengkap:} Transformasi Laplace sangat efektif
    karena mengubah persamaan diferensial linear menjadi persamaan
    aljabar, yang jauh lebih mudah dipecahkan.
  \end{itemize}
\item
  \textbf{Z-TRANSFORM-UBAH-PB-ALJABAR}

  \begin{itemize}
  \item
    \textbf{A}ktor: Transformasi Z
  \item
    \textbf{B}erperilaku: mengubah
  \item
    \textbf{C}ondisi: persamaan beda
  \item
    \textbf{D}erajat: menjadi persamaan aljabar, menyederhanakan
    penyelesaian
  \item
    \textbf{Pernyataan Lengkap:} Transformasi Z mengubah persamaan beda
    menjadi persamaan aljabar, menyederhanakan penyelesaian.
  \end{itemize}
\item
  \textbf{PETA-PENGETAHUAN-IMPLIKASI-PEMBELAJARAN}

  \begin{itemize}
  \item
    \textbf{A}ktor: Peta pengetahuan
  \item
    \textbf{B}erperilaku: memiliki implikasi besar
  \item
    \textbf{C}ondisi: (ketika digunakan)
  \item
    \textbf{D}erajat: untuk pembelajaran dan aplikasi praktis di bidang
    teknik
  \item
    \textbf{Pernyataan Lengkap:} Peta pengetahuan yang diuraikan dalam
    laporan ini memiliki implikasi besar untuk pembelajaran dan aplikasi
    praktis di bidang teknik.
  \end{itemize}
\item
  \textbf{PETA-MEMUNGKINKAN-GAMBARAN-BESAR}

  \begin{itemize}
  \item
    \textbf{A}ktor: Peta ini
  \item
    \textbf{B}erperilaku: memungkinkan
  \item
    \textbf{C}ondisi: individu (yang menggunakannya)
  \item
    \textbf{D}erajat: untuk ``melihat gambaran besar'' dan mengorganisir
    pengetahuan mereka dengan cara yang memfasilitasi pemahaman,
    pengambilan, dan aplikasi
  \item
    \textbf{Pernyataan Lengkap:} Peta ini memungkinkan individu untuk
    ``melihat gambaran besar'' dan mengorganisir pengetahuan mereka
    dengan cara yang memfasilitasi pemahaman, pengambilan, dan aplikasi.
  \end{itemize}
\item
  \textbf{AI-KURANGI-UPAYA-PETA}

  \begin{itemize}
  \item
    \textbf{A}ktor: AI
  \item
    \textbf{B}erperilaku: dapat mengurangi
  \item
    \textbf{C}ondisi: (ketika digunakan untuk) membuat peta pengetahuan
  \item
    \textbf{D}erajat: upaya manual yang diperlukan untuk membuat peta
    pengetahuan yang komprehensif dari buku teks, catatan kuliah, atau
    makalah penelitian
  \item
    \textbf{Pernyataan Lengkap:} AI dapat mengurangi upaya manual yang
    diperlukan untuk membuat peta pengetahuan yang komprehensif dari
    buku teks, catatan kuliah, atau makalah penelitian.
  \end{itemize}
\item
  \textbf{AI-PERSONALISASI-PETA-BELAJAR}

  \begin{itemize}
  \item
    \textbf{A}ktor: AI
  \item
    \textbf{B}erperilaku: dapat menyesuaikan
  \item
    \textbf{C}ondisi: kompleksitas dan fokus peta berdasarkan gaya
    belajar atau tingkat pemahaman individu
  \item
    \textbf{D}erajat: menciptakan jalur pembelajaran yang
    dipersonalisasi
  \item
    \textbf{Pernyataan Lengkap:} AI dapat menyesuaikan kompleksitas dan
    fokus peta berdasarkan gaya belajar atau tingkat pemahaman individu,
    menciptakan jalur pembelajaran yang dipersonalisasi.
  \end{itemize}
\item
  \textbf{PETA-AI-INTERAKTIF-IDENTIFIKASI-GAP}

  \begin{itemize}
  \item
    \textbf{A}ktor: Peta pengetahuan yang didukung AI
  \item
    \textbf{B}erperilaku: dapat menjadi interaktif
  \item
    \textbf{C}ondisi: (ketika digunakan)
  \item
    \textbf{D}erajat: memungkinkan pengguna untuk mengajukan pertanyaan
    dalam bahasa alami, menjelajahi hubungan, dan mengidentifikasi
    kesenjangan pengetahuan secara dinamis
  \item
    \textbf{Pernyataan Lengkap:} Peta pengetahuan yang didukung AI dapat
    menjadi interaktif, memungkinkan pengguna untuk mengajukan
    pertanyaan dalam bahasa alami, menjelajahi hubungan, dan
    mengidentifikasi kesenjangan pengetahuan secara dinamis.
  \end{itemize}
\item
  \textbf{PETA-AI-PANDU-SOLUSI}

  \begin{itemize}
  \item
    \textbf{A}ktor: Peta berbasis AI
  \item
    \textbf{B}erperilaku: dapat memandu
  \item
    \textbf{C}ondisi: pengguna (dalam pemecahan masalah)
  \item
    \textbf{D}erajat: melalui langkah-langkah pemecahan masalah,
    menyarankan formula atau transformasi yang relevan, dan bahkan
    membantu mengidentifikasi miskonsepsi umum
  \item
    \textbf{Pernyataan Lengkap:} Peta berbasis AI dapat memandu pengguna
    melalui langkah-langkah pemecahan masalah, menyarankan formula atau
    transformasi yang relevan, dan bahkan membantu mengidentifikasi
    miskonsepsi umum.
  \end{itemize}
\end{enumerate}

\bookmarksetup{startatroot}

\chapter{RPS Sinyal \& Sistem: Peta Pengetahuan \& Pemecahan
Masalah}\label{rps-sinyal-sistem-peta-pengetahuan-pemecahan-masalah}

(Saved responses are view only)

Berikut adalah Rencana Pembelajaran Semester (RPS) untuk mata kuliah EL
2007 Sinyal dan Sistem, yang telah diperbarui dengan mengintegrasikan
konsep Peta Pengetahuan dan kerangka Pemecahan Masalah ahli. Pendekatan
ini bertujuan untuk tidak hanya menyampaikan materi dasar sinyal dan
sistem, tetapi juga untuk melatih mahasiswa berpikir secara strategis
dan sistematis dalam analisis dan pemecahan masalah teknik.

\textbf{Ringkasan Pendekatan Pedagogis yang Diperbarui:}

RPS ini mengkonseptualisasikan setiap masalah teknik
sebagai~\textbf{``celah''}~antara informasi yang diketahui (titik awal)
dan hasil yang diinginkan (titik akhir). Pemecahan masalah kemudian
menjadi proses~\textbf{``menemukan rute dan kendaraan yang sesuai,
memilih rute terbaik, dan melintasinya''}~di dalam ``Peta Pengetahuan''.
Akan ada dua jenis Peta Pengetahuan yang digunakan dan dibangun oleh
mahasiswa:

1.~\textbf{Peta Pengetahuan Primitif (Primitive Knowledge Map)}: Ini
adalah representasi statis dari konsep-konsep inti, definisi, properti,
dan hubungan fundamental dalam Sinyal dan Sistem. Ini berfungsi sebagai
``ensiklopedia'' atau fondasi konseptual.

2.~\textbf{Peta Rute Pemecahan Masalah (Problem-Solving Knowledge Map)}:
Ini adalah alat yang dinamis dan berorientasi proses yang memandu
mahasiswa melalui langkah-langkah prosedural, titik keputusan, dan
``kendaraan'' (operasi, algoritma, heuristik) yang diperlukan untuk
menavigasi dari titik awal masalah ke solusinya.

Melalui pendekatan ini, mahasiswa didorong untuk mengembangkan pemahaman
konseptual yang lebih dalam, meningkatkan efisiensi pemecahan masalah,
dan menumbuhkan perilaku belajar yang teregulasi diri dan seperti ahli.
Alat digital akan dimanfaatkan untuk memfasilitasi pembuatan peta ini
secara interaktif dan kolaboratif.

\begin{center}\rule{0.5\linewidth}{0.5pt}\end{center}

\textbf{Rencana Pembelajaran Semester (RPS) EL 2007 Sinyal dan Sistem}

\textbf{Mata Kuliah:}~EL 2007 Sinyal dan Sistem~\textbf{Jumlah Jam
Sesi:}~39 Jam Sesi (diluar UTS dan UAS)~\textbf{Tujuan Umum:}~Mahasiswa
dapat menguasai materi dasar sinyal sistem untuk waktu kontinu (CT) dan
waktu diskrit (DT), termasuk deskripsi sinyal dan sistem dalam domain
waktu nyata, domain frekuensi, dan domain kompleks, serta mampu
menerapkan konsep-konsep ini dalam analisis dan pemecahan masalah
rekayasa.

\textbf{Struktur Pembelajaran Per Bab dan Integrasi Peta Pengetahuan:}

\textbf{Bab 1: Sinyal dan Sistem (Jam Sesi 1-3)}

•~\textbf{Tujuan Pembelajaran:}

~~~~◦~Peserta menyadari pentingnya konsep sinyal dan sistem dalam
konteks rekayasa dan dapat mendefinisikan sinyal dan sistem.

~~~~◦~Peserta mengenal sinyal dan sistem di alam dan memahami konsep
pemodelannya.

~~~~◦~Peserta mengenal ringkasan konsep Sinyal Sistem Waktu Diskrit dan
Waktu Kontinu.

~~~~◦~Peserta dapat memodelkan sinyal sebagai fungsi (transformasi
waktu, sinyal periodik/ganjil/genap, sinyal impuls/step).

•~\textbf{Integrasi Peta Pengetahuan:}

~~~~◦~\textbf{Pengantar Peta Pengetahuan:}~Perkenalkan
konsep~\textbf{Peta Pengetahuan (Knowledge Map)}~sebagai alat navigasi
untuk memahami domain Sinyal dan Sistem secara visual, menyoroti
interkonektivitas konsep dan mencegah ``silo pengetahuan''.

~~~~◦~\textbf{Metafora Pemecahan Masalah:}~Jelaskan
metafora~\textbf{``masalah sebagai celah''}~antara apa yang diketahui
(Titik Mulai) dan apa yang dicari (Titik Akhir),
serta~\textbf{``pemecahan masalah sebagai pencarian rute dengan
kendaraan''}.

~~~~◦~\textbf{Peta Pengetahuan Primitif (Awal):}~Mulai
identifikasi~\textbf{Titik (Nodes)}~kunci: ``Sinyal,'' ``Sistem,''
``Domain Waktu,'' ``Domain Frekuensi,'' ``Domain Kompleks''. Diskusikan
definisi dasar sinyal sebagai representasi fenomena fisik pembawa
informasi, dan sistem sebagai entitas pemroses input menjadi output.

~~~~◦~\textbf{Hubungan Input-Sistem-Output:}~Jelaskan tiga masalah dasar
dalam analisis sistem (mencari output, mencari sistem, mencari input)
sebagai inti pemecahan masalah.

~~~~◦~\textbf{Kendaraan Dasar:}~Kenalkan operasi dasar sinyal
(pergeseran, penskalaan, pembalikan waktu; penskalaan amplitudo,
penjumlahan, perkalian) sebagai ``Kendaraan Matematika Fundamental''
(K\_MAT\_) awal.

\textbf{Bab 2: Sistem LTI di Kawasan Waktu (Jam Sesi 4-9)}

•~\textbf{Tujuan Pembelajaran:}

~~~~◦~Peserta dapat memodelkan sistem (waktu diskrit/kontinu) sebagai
persamaan input/output dan menghitung respons sistem.

~~~~◦~Peserta mengenali jenis-jenis sistem berdasarkan perilaku/sifat
dasarnya (linearitas, invariansi waktu, kausalitas, stabilitas, memori,
invertibilitas), serta dapat memeriksanya.

~~~~◦~Peserta dapat merepresentasikan sistem LTI dengan respons impuls
dan menghitung output sistem LTI dengan konvolusi.

~~~~◦~Peserta dapat menerapkan model persamaan diferensial/beda (LCCDE)
dan menggambarkan diagram blok sistem.

•~\textbf{Integrasi Peta Pengetahuan:}

~~~~◦~\textbf{Peta Rute Pemecahan Masalah:}~Fokus pada masalah ``Mencari
Output'' di Domain Waktu. Identifikasi sinyal input (Titik Mulai) dan
respons impuls sistem, lalu output (Titik Akhir).

~~~~◦~\textbf{Kendaraan Konvolusi:}~Perkenalkan~\textbf{Konvolusi
Integral (untuk CT)}~dan~\textbf{Konvolusi Sum (untuk DT)}~sebagai
``Kendaraan'' utama untuk menghitung output sistem LTI di domain waktu.
Tekankan bagaimana konvolusi merepresentasikan interaksi input dan
respons impuls.

~~~~◦~\textbf{Sifat Sistem LTI sebagai Rute:}~Linearitas (prinsip
superposisi) dan invariansi waktu adalah kunci yang memungkinkan
penggunaan konvolusi. Bahas kausalitas dan stabilitas BIBO sebagai
properti penting yang menentukan perilaku sistem.

~~~~◦~\textbf{Kendaraan Visual:}~Gunakan~\textbf{Diagram Blok}~sebagai
``Kendaraan Visual'' (K\_VIS\_) untuk merepresentasikan sistem LCCDE dan
aliran sinyal.

~~~~◦~\textbf{Heuristik Awal:}~Perkenalkan heuristik ``Gambar Diagram''
sebagai alat bantu visualisasi masalah. Jelaskan mengapa dekonvolusi di
domain waktu sulit dan menjadi motivasi untuk transformasi.

\textbf{Bab 3: Deret Fourier untuk Sinyal Periodik (Jam Sesi 10-15)}

•~\textbf{Tujuan Pembelajaran:}

~~~~◦~Peserta memahami model deret Fourier dari sinyal periodik (CT/DT)
dan konsep fungsi eigen dalam bentuk eksponensial kompleks.

~~~~◦~Peserta dapat mendefinisikan deret Fourier dan menerapkannya untuk
merepresentasikan sinyal periodik, serta memahami konsep kandungan
frekuensi.

~~~~◦~Peserta mengetahui berbagai sifat deret Fourier dan menerapkannya.

~~~~◦~Peserta dapat menerapkan deret Fourier untuk menentukan respons
sinyal periodik pada sistem LTI (sebagai filter).

•~\textbf{Integrasi Peta Pengetahuan:}

~~~~◦~\textbf{Transformasi sebagai Jembatan/Kendaraan:}~Perkenalkan
Deret Fourier Waktu Kontinu (CTFS) sebagai~\textbf{``Kendaraan
Transformasi''}~(TRANSFORMASI-JEMBATAN-DOMAIN-PERUBAHAN) yang mengubah
sinyal periodik dari Domain Waktu ke Domain Frekuensi.

~~~~◦~\textbf{Kandungan Frekuensi:}~Tekankan bagaimana analisis di
Domain Frekuensi memberikan wawasan tentang~\textbf{``kandungan
frekuensi''}~sinyal, membantu memahami spektrum sinyal periodik.

~~~~◦~\textbf{Respons Frekuensi:}~Hubungkan ``Respons Frekuensi''
(DF\_ResponsFrekuensi) sebagai properti penting sistem LTI di domain
frekuensi. Ini adalah ``Kendaraan Diagnostik'' untuk memahami bagaimana
sistem memodifikasi komponen frekuensi.

~~~~◦~\textbf{Aplikasi Filtering:}~Bahas bagaimana sistem LTI dapat
berfungsi sebagai ``Filter'' (mis. filter low-pass, high-pass) di domain
frekuensi.

\textbf{Bab 4: Transformasi Fourier Waktu Kontinu (Jam Sesi 16-18)}

•~\textbf{Tujuan Pembelajaran:}

~~~~◦~Peserta dapat memodelkan sinyal CT menggunakan Transformasi
Fourier, termasuk sinyal aperiodik dan periodik, serta memastikan
konvergensinya.

~~~~◦~Peserta memahami dan membuktikan berbagai sifat transformasi
Fourier (CTFT) dan menggunakannya untuk menentukan transformasi sinyal.

~~~~◦~Peserta dapat memodelkan sistem LCCDE menggunakan Transformasi
Fourier dan menerapkannya untuk menghitung respons sistem.

•~\textbf{Integrasi Peta Pengetahuan:}

~~~~◦~\textbf{CTFT sebagai Kendaraan Utama:}~Tegaskan Transformasi
Fourier Waktu Kontinu (CTFT) sebagai~\textbf{``Kendaraan
Transformasi''}~(CTFT-UBAH-WAKTU-FREKUENSI) yang mengubah sinyal domain
waktu menjadi representasi domain frekuensi kontinu.

~~~~◦~\textbf{Penyederhanaan Konvolusi:}~Tekankan properti kunci CTFT
yang mengubah~\textbf{operasi konvolusi di domain waktu menjadi
perkalian sederhana di domain frekuensi}~(CTFT-KONVOLUSI-PERKALIAN). Ini
adalah ``Kendaraan Penyederhanaan'' yang krusial
(TRANSFORMASI-TUJUAN-PENYEDERHANAAN).

~~~~◦~\textbf{Peta Rute Pemecahan Masalah:}~Latih mahasiswa menggunakan
rute: Input x(t) (DW\_SinyalDasar) -\textgreater{} CTFT
(REPRESENTASI\_FREK\_DARI) -\textgreater{} X(jω) (DF\_TransFourier).
Output y(t) (DW\_SinyalDasar) -\textgreater{} Y(jω) = X(jω)H(jω)
(OUTPUT-FREKUENSI-CTFT-PERKALIAN).

\textbf{Bab 5: Transformasi Fourier Waktu Diskrit (Jam Sesi 19-21)}

•~\textbf{Tujuan Pembelajaran:}

~~~~◦~Peserta dapat memodelkan sinyal DT menggunakan Transformasi
Fourier, termasuk sinyal aperiodik dan periodik, serta memastikan
konvergensinya.

~~~~◦~Peserta memahami dan membuktikan berbagai sifat transformasi
Fourier (DTFT/DFT) dan menggunakannya.

~~~~◦~Peserta dapat memodelkan sistem LCCDE menggunakan Transformasi
Fourier dan menerapkannya untuk menghitung respons sistem.

•~\textbf{Integrasi Peta Pengetahuan:}

~~~~◦~\textbf{DTFT/DFT sebagai Analog Diskrit:}~Perkenalkan Transformasi
Fourier Waktu Diskrit (DTFT) dan Discrete Fourier Transform (DFT)
sebagai analog diskrit dari CTFT.

~~~~◦~\textbf{Penyederhanaan di Domain Frekuensi Diskrit:}~Sama seperti
CTFT, DTFT/DFT juga mengubah konvolusi menjadi perkalian di domain
frekuensi, menjadikannya ``Kendaraan Efisien'' untuk analisis sistem LTI
waktu diskrit.

\textbf{Bab 6: Filter dan Karakterisasi Waktu-Frekuensi (Jam Sesi
22-24)}

•~\textbf{Tujuan Pembelajaran:}

~~~~◦~Peserta dapat menentukan representasi magnituda-fasa dari
sistem/filter.

~~~~◦~Peserta dapat memahami sifat filter sebagai pengubah frekuensi
secara selektif, serta membedakan filter ideal dan tidak ideal.

~~~~◦~Peserta mengenali filter (CT dan DT) berorde rendah, serta sifat
waktu-frekuensi nya.

•~\textbf{Integrasi Peta Pengetahuan:}

~~~~◦~\textbf{Filter sebagai Aplikasi Kunci:}~Fokus pada ``Filter''
(Aplikasi) sebagai manifestasi praktis dari pemahaman sinyal dan sistem.

~~~~◦~\textbf{Kendaraan Visual untuk
Karakterisasi:}~Gunakan~\textbf{Plot Respons Frekuensi (Magnituda dan
Fasa)}~dan~\textbf{Bode Plot}~(K\_VIS\_BodePlot) sebagai ``Kendaraan
Visual''~untuk menganalisis karakteristik filter dan sistem di domain
frekuensi.

~~~~◦~\textbf{Heuristik:}~``Mencari Pola'' dalam respons frekuensi untuk
mengidentifikasi jenis filter (mis. low-pass, high-pass, band-pass).

\textbf{Bab 7: Sampling (Jam Sesi 25-27)}

•~\textbf{Tujuan Pembelajaran:}

~~~~◦~Peserta dapat merepresentasikan sinyal CT menggunakan sinyal DT
melalui sampling.

~~~~◦~Peserta memahami konsep aliasing serta cara menghindarinya
(Teorema Nyquist-Shannon).

~~~~◦~Peserta dapat memanfaatkan sampling untuk memproses sinyal CT
menggunakan sistem DT.

•~\textbf{Integrasi Peta Pengetahuan:}

~~~~◦~\textbf{Sampling sebagai Jembatan
Domain:}~Perkenalkan~\textbf{Sampling}~(AD\_Sampling) sebagai ``konsep
fundamental yang menjembatani sinyal waktu kontinu (analog) dengan
sinyal waktu diskrit (digital)''.

~~~~◦~\textbf{Kendaraan Penentu Aliasing:}~Tekankan~\textbf{Teorema
Nyquist-Shannon}~(SAMPLING-HINDARI-ALIASING-NYQUIST) sebagai ``Kendaraan
Penentu'' untuk menghindari distorsi aliasing.

~~~~◦~\textbf{Rekonstruksi:}~Bahas ``Rekonstruksi Sinyal'' sebagai
proses invers dari sampling, menggunakan filter low-pass ideal. Ini
adalah contoh ``Bekerja Mundur'' dalam pemecahan masalah.

\textbf{Bab 8: Transformasi Laplace (Jam Sesi 28-33)}

•~\textbf{Tujuan Pembelajaran:}

~~~~◦~Peserta dapat merepresentasikan sinyal CT dengan transformasi
Laplace, mendefinisikan, menghitung, dan memastikan konvergensinya
(ROC).

~~~~◦~Peserta memahami konsep pole-zero, menghitungnya, serta melakukan
evaluasi sinyal/sistem secara geometri.

~~~~◦~Peserta memahami sifat transformasi dan tabel pasangan, serta
proses inversi (termasuk partial fraction).

~~~~◦~Peserta dapat memodelkan sistem LTI dan LCCDE CT dengan
transformasi Laplace (fungsi sistem, pole-zero sistem, diagram blok,
unilateral Laplace Transform).

•~\textbf{Integrasi Peta Pengetahuan:}

~~~~◦~\textbf{Transformasi Laplace sebagai Kendaraan
Generalisasi:}~Posisikan Transformasi Laplace (DK\_TransLaplace) sebagai
``Kendaraan Generalisasi'' (GENERALISASI\_DARI) dari Transformasi
Fourier, yang sangat efektif untuk menganalisis sistem LTI dan
menyelesaikan persamaan diferensial (LAPLACE-EFEKTIF-PD-ALJABAR).

~~~~◦~\textbf{Analisis Pole-Zero \& ROC sebagai Kendaraan
Diagnostik:}~Tekankan~\textbf{Analisis Pole-Zero
(DK\_PoleZeroPlot)}~dan~\textbf{Region of Convergence (ROC)}~sebagai
``Kendaraan Diagnostik'' (MENENTUKAN\_STABILITAS) untuk memahami
stabilitas (SISTEM-STABIL-POLE-KIRI) dan kausalitas sistem.

~~~~◦~\textbf{Peta Rute Pemecahan Masalah:}~Contoh rute masalah:
Diberikan persamaan diferensial (Titik Mulai) -\textgreater{} Terapkan
Transformasi Laplace (Kendaraan) -\textgreater{} Dapatkan Fungsi
Transfer H(s) -\textgreater{} Analisis Pole-Zero (Kendaraan Visual)
-\textgreater{} Tentukan Stabilitas (Titik Keputusan) -\textgreater{}
Inversi Transformasi Laplace (Kendaraan) untuk mendapatkan respons
impuls h(t) (Titik Akhir).

\textbf{Bab 9: Transformasi Z (Jam Sesi 34-39)}

•~\textbf{Tujuan Pembelajaran:}

~~~~◦~Peserta dapat merepresentasikan sinyal DT dengan transformasi Z,
mendefinisikan, menghitung, dan memastikan konvergensinya (ROC).

~~~~◦~Peserta memahami konsep pole-zero, menghitungnya, serta melakukan
evaluasi sinyal/sistem secara geometri.

~~~~◦~Peserta memahami sifat transformasi dan tabel pasangan, serta
proses inversi (termasuk partial fraction).

~~~~◦~Peserta dapat memodelkan sistem LTI dan LCCDE DT dengan
transformasi Z (fungsi sistem, pole-zero sistem, diagram blok,
unilateral Z-Transform).

•~\textbf{Integrasi Peta Pengetahuan:}

~~~~◦~\textbf{Transformasi Z sebagai Analog Diskrit:}~Perkenalkan
Transformasi Z (DK\_TransZ) sebagai analog diskrit dari Transformasi
Laplace (ANALOG\_DT\_DARI), yang penting untuk analisis sistem LTI waktu
diskrit dan penyelesaian persamaan beda (Z-TRANSFORM-UBAH-PB-ALJABAR).

~~~~◦~\textbf{Analisis Pole-Zero \& ROC (DT):}~Tekankan kembali Analisis
Pole-Zero dan ROC untuk sistem waktu diskrit, dengan kondisi stabilitas
bahwa~\textbf{semua pole harus berada di dalam lingkaran unit di
z-plane}~(SISTEM-STABIL-POLE-UNIT-LINGKARAN).

~~~~◦~\textbf{Peta Rute Pemecahan Masalah:}~Contoh rute masalah:
Diberikan persamaan beda (Titik Mulai) -\textgreater{} Terapkan
Transformasi Z (Kendaraan) -\textgreater{} Dapatkan Fungsi Transfer H(z)
-\textgreater{} Analisis Pole-Zero (Kendaraan Visual) -\textgreater{}
Tentukan Stabilitas (Titik Keputusan) -\textgreater{} Inversi
Transformasi Z (Kendaraan) untuk mendapatkan respons impuls h{[}n{]}
(Titik Akhir).

\begin{center}\rule{0.5\linewidth}{0.5pt}\end{center}

\textbf{Strategi Pedagogis Tambahan untuk Implementasi Peta
Pengetahuan:}

•~\textbf{Pemodelan dan Scaffolding:}~Instruktur akan secara eksplisit
memodelkan proses pembuatan Peta Pengetahuan dan penggunaan Peta Rute
Pemecahan Masalah di kelas, mengartikulasikan proses berpikir mereka.
Mahasiswa akan diberikan peta yang sebagian sudah selesai atau bank kata
kunci pada awalnya, dengan tingkat kesulitan yang ditingkatkan secara
bertahap.

•~\textbf{Konstruksi Peta Kolaboratif:}~Mahasiswa didorong untuk
berkolaborasi dalam membuat dan merevisi peta pengetahuan mereka. Alat
digital yang mendukung kolaborasi~\emph{real-time}~seperti Miro,
MindMeister, atau Microsoft Visio akan direkomendasikan untuk
memfasilitasi proses ini, membantu tim mengembangkan model mental
bersama tentang sistem yang kompleks.

•~\textbf{Portfolio Kuliah sebagai ``Dokumen Hidup'':}~Portfolio akan
berisi berbagai dokumen karya hasil belajar dan tugas, yang ditautkan di
blog pribadi mahasiswa. Mahasiswa didorong untuk melihat peta mereka
bukan sebagai tugas statis, tetapi sebagai~\textbf{``dokumen
hidup''}~yang dapat terus ditinjau dan disempurnakan seiring
berkembangnya pemahaman. Ini menumbuhkan kesadaran metakognitif dan
regulasi diri.

•~\textbf{Penilaian Berbasis Peta:}~Peta pengetahuan akan digunakan
sebagai alat penilaian formatif dan sumatif, memberikan ``gambaran jelas
tentang pemahaman konseptual siswa''. Mahasiswa mungkin diminta untuk
menulis refleksi tertulis tentang rute yang mereka pilih dan alasannya
dalam pemecahan masalah.

•~\textbf{Pemanfaatan Alat Digital untuk Mengurangi Beban
Kognitif:}~Penggunaan alat digital akan mengurangi ``beban kognitif
ekstrinsik'' yang terkait dengan menggambar peta, memungkinkan siswa
untuk lebih berkonsentrasi pada~\textbf{konten dan hubungan antar
konsep}. Fitur seperti notasi matematika di Math Whiteboard juga akan
dimanfaatkan.

•~\textbf{Peran Kecerdasan Buatan (AI):}~Diskusikan bagaimana AI,
seperti~\textbf{NotebookLM}, dapat bertindak sebagai asisten riset
pribadi yang memproses sumber multimodal (teks, video, audio),
memberikan wawasan instan, dan menjelaskan konsep kompleks dengan
kutipan yang jelas. AI juga dapat mendukung pembuatan peta konsep
otomatis dan personalisasi jalur pembelajaran. Ini akan mengubah peran
komputasi dalam kurikulum, menjadikannya kendaraan integral untuk
eksplorasi.

Pendekatan ini diharapkan dapat memberdayakan mahasiswa untuk tidak
hanya memahami materi Sinyal dan Sistem, tetapi juga
untuk~\textbf{berpikir seperti insinyur --- secara strategis dan
sistematis}.

\bookmarksetup{startatroot}

\chapter{Merancang Sistem Asesmen Pembelajaran Berbasis Nilai
Pengetahuan di Bidang Sinyal dan
Sistem}\label{merancang-sistem-asesmen-pembelajaran-berbasis-nilai-pengetahuan-di-bidang-sinyal-dan-sistem}

\section{I. Pendahuluan: Merancang Ekosistem Pembelajaran Berbasis Nilai
di Sinyal dan
Sistem}\label{i.-pendahuluan-merancang-ekosistem-pembelajaran-berbasis-nilai-di-sinyal-dan-sistem}

\section{A. Latar Belakang: Tantangan Asesmen Tradisional dan Potensi
Inovasi}\label{a.-latar-belakang-tantangan-asesmen-tradisional-dan-potensi-inovasi}

Asesmen tradisional dalam pendidikan tinggi, terutama di bidang teknik
yang kompleks seperti Sinyal dan Sistem, seringkali berfokus pada
hafalan dan kemampuan mahasiswa untuk mereproduksi informasi. Pendekatan
ini, yang umumnya melibatkan ujian tengah semester dan ujian akhir,
dapat menimbulkan tekanan dan kecemasan yang signifikan pada mahasiswa,
yang pada gilirannya dapat menghambat pemahaman materi secara mendalam.1
Penekanan pada hasil akhir tunggal ini seringkali mengabaikan proses
pembelajaran dan perkembangan kognitif mahasiswa sepanjang semester.

Dalam konteks disiplin ilmu Sinyal dan Sistem, di mana pemahaman
konseptual dan kemampuan pemecahan masalah sangat krusial, metode
asesmen yang lebih inovatif diperlukan. Tujuannya adalah untuk mendorong
keterlibatan aktif mahasiswa, memotivasi mereka untuk mengeksplorasi dan
menguasai materi secara lebih mendalam, serta memfasilitasi pengembangan
keterampilan berpikir tingkat tinggi. Sistem asesmen yang dirancang
secara strategis dapat bertindak sebagai alat pedagogis yang kuat, bukan
hanya sebagai mekanisme evaluasi semata.

\section{B. Tujuan Sistem Asesmen Berbasis Nilai
Pengetahuan}\label{b.-tujuan-sistem-asesmen-berbasis-nilai-pengetahuan}

Sistem asesmen yang diusulkan ini dirancang dengan beberapa tujuan
fundamental:

\begin{enumerate}
\def\labelenumi{\arabic{enumi}.}
\item
  \textbf{Meningkatkan Keterlibatan Mahasiswa:} Dengan memperkenalkan
  insentif yang jelas dan pengakuan atas kontribusi pengetahuan, sistem
  ini bertujuan untuk secara signifikan meningkatkan motivasi dan
  partisipasi aktif mahasiswa dalam proses pembelajaran.
\item
  \textbf{Memberikan ``Sense of Value Creation'':} Mahasiswa akan
  merasakan bahwa karya intelektual mereka memiliki nilai nyata dan
  dapat dimanfaatkan oleh orang lain, tidak hanya sekadar tugas yang
  dinilai dan kemudian dilupakan. Ini menumbuhkan rasa kepemilikan dan
  kebanggaan terhadap hasil belajar mereka.2
\item
  \textbf{Memfasilitasi Pelacakan Penguasaan Topik dan Tingkat
  Kognitif:} Sistem ini memungkinkan pelacakan yang lebih granular
  terhadap topik materi yang telah dikuasai mahasiswa dan tingkat
  kedalaman pemahaman mereka sesuai dengan Taksonomi Bloom.
\item
  \textbf{Membangun Repositori Pengetahuan Dinamis:} Karya-karya terbaik
  mahasiswa akan membentuk sebuah basis pengetahuan yang terus
  berkembang, relevan, dan kontekstual, yang dihasilkan oleh komunitas
  pembelajar itu sendiri.
\end{enumerate}

\section{C. Gambaran Umum Sistem ``Knowledge
Marketplace''}\label{c.-gambaran-umum-sistem-knowledge-marketplace}

Untuk mencapai tujuan-tujuan tersebut, sistem ini mengusulkan sebuah
model ``Knowledge Marketplace'' yang dinamis dan terintegrasi. Setiap
minggu, dosen akan ``mengiklankan'' kebutuhan akan ``karya pengetahuan
dan pemecahan masalah'' tertentu. Mahasiswa kemudian akan menghasilkan
dokumen laporan, khususnya dalam bentuk peta pengetahuan (knowledge
maps), yang merepresentasikan pemahaman mereka terhadap topik atau
solusi masalah. Dokumen-dokumen ini akan ``dibeli'' oleh dosen
menggunakan sistem mata uang digital berjenjang: Point Uang, Point Emas,
Point Platinum, dan Point Berlian. Nilai pembelian ini akan dikaitkan
secara langsung dengan tingkat kognitif Taksonomi Bloom yang dicapai
oleh karya tersebut dan domain teknis spesifik dalam Sinyal dan Sistem
yang dibahas.

Lebih lanjut, sistem ini mengintegrasikan mata uang fiat yang berbeda
(misalnya, IDR, MYR, USD, AUD, GBP, EUR, JPY, KRW, CNY) untuk setiap
domain teknis, menambah lapisan insentif dan simulasi ekonomi.
Karya-karya yang telah ``dibeli'' akan diunggah ke website kuliah,
menjadi sumber belajar yang berharga bagi mahasiswa di tahun berikutnya.
Di akhir perkuliahan, total ``harta'' (akumulasi poin dan mata uang
fiat) yang terkumpul oleh setiap mahasiswa akan diindeks untuk
mendapatkan nilai akhir mata kuliah. Model ini menciptakan siklus
pembelajaran berkelanjutan dan mendorong mahasiswa untuk tidak hanya
belajar, tetapi juga berkontribusi pada ekosistem pengetahuan.

\section{II. Fondasi Pedagogis: Taksonomi Bloom dan Peta Pengetahuan
untuk Penguasaan
Konsep}\label{ii.-fondasi-pedagogis-taksonomi-bloom-dan-peta-pengetahuan-untuk-penguasaan-konsep}

\section{A. Memahami Tingkat Kognitif Taksonomi Bloom dalam Pembelajaran
Sinyal dan
Sistem}\label{a.-memahami-tingkat-kognitif-taksonomi-bloom-dalam-pembelajaran-sinyal-dan-sistem}

Taksonomi Bloom yang direvisi menyediakan kerangka kerja yang sistematis
untuk mengklasifikasikan tujuan pembelajaran kognitif. Kerangka ini
mengidentifikasi enam tingkatan keterampilan berpikir, yang bergerak
dari yang paling sederhana hingga yang paling kompleks: Mengingat
(Remember), Memahami (Understand), Menerapkan (Apply), Menganalisis
(Analyze), Mengevaluasi (Evaluate), dan Menciptakan (Create).4 Setiap
tingkatan ini memiliki karakteristik konseptual yang berbeda dan
memerlukan jenis aktivitas kognitif yang spesifik.

Penggunaan kata kerja tindakan yang tepat dan terukur sangat penting
dalam merumuskan tujuan pembelajaran dan merancang asesmen yang selaras
dengan setiap tingkatan Bloom.4 Misalnya, kata kerja seperti
``mendefinisikan'' atau ``membuat daftar'' cocok untuk tingkat
Mengingat, sementara ``menganalisis'' atau ``merancang'' sesuai untuk
tingkat yang lebih tinggi.

Sifat hierarkis Taksonomi Bloom, di mana penguasaan pada tingkat yang
lebih tinggi bergantung pada pencapaian pengetahuan dan keterampilan
prasyarat pada tingkat yang lebih rendah, memiliki dampak langsung pada
desain asesmen. Untuk dapat ``menerapkan'' suatu konsep, mahasiswa harus
terlebih dahulu ``memahami'' konsep tersebut.8 Demikian pula, untuk
``mengevaluasi'' suatu proses, mahasiswa harus terlebih dahulu
``menganalisisnya''.8 Oleh karena itu, tugas asesmen harus dirancang
secara cermat untuk menargetkan keterampilan kognitif yang spesifik pada
setiap level, memastikan validitas penilaian penguasaan materi.

\subsection{1. Level Dasar: Mengingat dan Memahami (Peta Pengetahuan
Primitif)}\label{level-dasar-mengingat-dan-memahami-peta-pengetahuan-primitif}

Pada level dasar Taksonomi Bloom, fokusnya adalah pada akuisisi dan
pemahaman informasi fundamental.

\begin{itemize}
\item
  \textbf{Mengingat:} Tingkat ini melibatkan kemampuan untuk mengambil,
  memanggil kembali, atau mengenali pengetahuan yang relevan dari memori
  jangka panjang. Contoh kata kerja yang sesuai meliputi: menyebutkan,
  mendefinisikan, menjelaskan, mengidentifikasi, dan membuat daftar.4
  Dalam konteks Sinyal dan Sistem, ini berarti mahasiswa dapat mengingat
  definisi sinyal waktu kontinu, sinyal waktu diskrit, atau properti
  dasar sistem.
\item
  \textbf{Memahami:} Tingkat ini menuntut mahasiswa untuk
  mendemonstrasikan pemahaman melalui berbagai bentuk penjelasan. Kata
  kerja yang relevan termasuk: menguraikan, menjelaskan, meringkas,
  mengklasifikasikan, dan membandingkan.4 Pada level ini, mahasiswa
  diharapkan dapat menjelaskan perbedaan antara sinyal periodik dan
  aperiodik, atau meringkas karakteristik sistem linear.
\end{itemize}

\textbf{Peta Pengetahuan Primitif} akan secara khusus dirancang untuk
menguji kemampuan pada dua level ini. Peta ini akan berfokus pada
representasi dasar konsep, definisi, dan karakteristik sinyal (seperti
sinyal waktu kontinu dan diskrit, sinyal periodik dan aperiodik, sinyal
energi dan daya) dan sistem (seperti linearitas, invarian waktu,
kausalitas, dan stabilitas).9 Peta ini akan menjadi fondasi untuk
membangun pemahaman yang lebih kompleks.

\subsection{2. Level Menengah: Menerapkan dan Menganalisis (Peta
Pengetahuan
Aplikatif)}\label{level-menengah-menerapkan-dan-menganalisis-peta-pengetahuan-aplikatif}

Level menengah melibatkan penggunaan pengetahuan dalam situasi baru dan
pemecahan masalah.

\begin{itemize}
\item
  \textbf{Menerapkan:} Ini adalah kemampuan untuk menggunakan informasi
  atau keterampilan dalam situasi yang belum pernah ditemui sebelumnya.
  Kata kerja aksi yang sesuai meliputi: menerapkan, menghitung,
  melaksanakan, memecahkan, dan menunjukkan.4 Contohnya adalah
  menghitung respons sistem terhadap input tertentu.
\item
  \textbf{Menganalisis:} Tingkat ini mengharuskan mahasiswa untuk
  memecah materi atau konsep menjadi bagian-bagian penyusunnya dan
  menentukan bagaimana bagian-bagian tersebut saling berhubungan atau
  dengan struktur keseluruhan. Kata kerja yang relevan termasuk:
  menganalisis, memecah, membandingkan, dan membedakan.4 Contohnya
  adalah menganalisis spektrum frekuensi sinyal atau membedakan properti
  berbagai transformasi.
\end{itemize}

\textbf{Peta Pengetahuan Aplikatif} akan menunjukkan bagaimana
konsep-konsep dasar diterapkan untuk memecahkan masalah, menganalisis
respons sistem, atau membedakan antara jenis sinyal atau properti
transformasi yang berbeda. Untuk mencapai tingkat ``menerapkan'' dan
``menganalisis'' dalam pendidikan teknik, mahasiswa perlu terlibat dalam
pemecahan masalah yang aktif. Pemecahan masalah dalam Sinyal dan Sistem
seringkali melibatkan penerapan transformasi matematis (seperti
Transformasi Fourier, Laplace, atau Z-Transform) atau operasi konvolusi.
Misalnya, penyelesaian persamaan diferensial menggunakan Transformasi
Laplace melibatkan serangkaian langkah spesifik: mentransformasikan
masalah ke dalam persamaan aljabar, melakukan manipulasi aljabar, dan
kemudian melakukan transformasi balik untuk mendapatkan solusi di domain
waktu.21 Demikian pula, operasi konvolusi, baik untuk sinyal waktu
kontinu maupun diskrit, memiliki prosedur yang terdefinisi untuk mencari
respons sistem.25 Oleh karena itu, peta pengetahuan aplikatif harus
secara eksplisit mendemonstrasikan langkah-langkah prosedural ini, bukan
hanya hasil akhirnya. Peta ini dapat berupa

\emph{flowchart} atau \emph{diagram alir} yang menggambarkan proses
pemecahan masalah, langkah-langkah algoritma, atau alur keputusan.29 Ini
secara langsung menghubungkan metode asesmen dengan keterampilan
pemecahan masalah inti dalam domain Sinyal dan Sistem.

\subsection{3. Level Lanjut: Mengevaluasi dan Menciptakan (Peta
Pengetahuan Aplikatif
Lanjutan)}\label{level-lanjut-mengevaluasi-dan-menciptakan-peta-pengetahuan-aplikatif-lanjutan}

Level tertinggi Taksonomi Bloom berfokus pada penilaian kritis dan
inovasi.

\begin{itemize}
\item
  \textbf{Mengevaluasi:} Kemampuan untuk membuat penilaian berdasarkan
  kriteria dan standar yang diberikan. Kata kerja aksi yang sesuai
  meliputi: menilai, mendukung, membandingkan, mengkritik, menentukan,
  dan mengevaluasi.4 Contohnya adalah mengevaluasi efisiensi dua
  algoritma pemrosesan sinyal yang berbeda.
\item
  \textbf{Menciptakan:} Ini adalah kemampuan untuk menggabungkan
  elemen-elemen untuk membentuk keseluruhan yang baru dan koheren, atau
  menata ulang elemen-elemen menjadi pola atau struktur baru. Kata kerja
  yang relevan meliputi: menyusun, membangun, merancang, mengembangkan,
  merumuskan, dan menciptakan.4 Contohnya adalah merancang filter
  digital baru atau mengembangkan pendekatan novel untuk analisis
  sinyal.
\end{itemize}

\textbf{Peta Pengetahuan Aplikatif Lanjutan} pada level ini dapat
melibatkan perancangan sistem atau algoritma pemrosesan sinyal yang
inovatif, atau evaluasi kritis terhadap sistem yang ada, menunjukkan
sintesis pengetahuan yang kompleks dan pemikiran orisinal.

\section{B. Peran Peta Pengetahuan dalam Mengukur dan Memvisualisasikan
Penguasaan}\label{b.-peran-peta-pengetahuan-dalam-mengukur-dan-memvisualisasikan-penguasaan}

Peta konsep, atau peta pengetahuan, merupakan alat pedagogis yang sangat
efektif yang membantu mahasiswa dalam menyusun pembelajaran mereka. Alat
ini memungkinkan mahasiswa untuk melihat ``gambaran besar'' dari suatu
topik, mengorganisir pengetahuan mereka, dan secara eksplisit
menunjukkan hubungan antar konsep.34 Peta pengetahuan dapat
merepresentasikan baik pengetahuan konseptual (fakta, konsep, dan objek)
maupun pengetahuan prosedural (kemampuan untuk melakukan sesuatu,
seperti langkah-langkah untuk memecahkan masalah).40 Integrasi kedua
jenis pengetahuan ini sangat penting untuk kinerja pemecahan masalah
yang efektif, karena tindakan tidak dapat dilakukan tanpa kesadaran akan
informasi konseptual yang diperlukan.40

Peta pengetahuan secara visual merepresentasikan pemahaman mahasiswa dan
koneksi yang mereka buat antar konsep.35 Jika seorang mahasiswa
kesulitan menghubungkan konsep-konsep tertentu atau menghilangkan
hubungan kunci, hal ini secara langsung menyoroti adanya kesenjangan
dalam pemahaman mereka. Kemampuan diagnostik visual ini lebih efisien
daripada meninjau jawaban teks tradisional dan memberikan umpan balik
yang dapat ditindaklanjuti baik untuk mahasiswa maupun instruktur.44 Ini
secara langsung mendukung tujuan dosen untuk ``melacak topik materi apa
yang sudah di kuasai dan mana yang belum.'' Dengan melihat struktur dan
kelengkapan peta, dosen dapat dengan cepat mengidentifikasi area yang
memerlukan intervensi atau penguatan lebih lanjut, memungkinkan
pembelajaran yang lebih personal dan adaptif.

\subsection{1. Peta Pengetahuan Konseptual untuk Pemahaman
Struktural}\label{peta-pengetahuan-konseptual-untuk-pemahaman-struktural}

Peta pengetahuan konseptual berfokus pada representasi definisi,
klasifikasi, properti, dan hubungan hierarkis antar konsep dalam domain
Sinyal dan Sistem. Ini membantu mahasiswa membangun kerangka kognitif
yang kohesif, di mana mereka dapat melihat bagaimana ide-ide besar
terurai menjadi sub-konsep dan bagaimana semuanya saling terkait.

\subsection{2. Peta Pengetahuan Prosedural untuk Pemecahan
Masalah}\label{peta-pengetahuan-prosedural-untuk-pemecahan-masalah}

Peta pengetahuan prosedural berfokus pada langkah-langkah, algoritma,
dan alur keputusan yang terlibat dalam memecahkan masalah teknis. Ini
bisa mencakup langkah-langkah untuk mencari fungsi transfer sistem,
proses dekonvolusi sinyal, atau tahapan dalam proses sampling.
Representasi ini paling efektif dalam bentuk \emph{flowchart} atau
\emph{diagram alir} yang secara visual memetakan urutan tindakan dan
keputusan.29

\section{C. Manfaat Peta Pengetahuan untuk Pembelajaran Mendalam dan
Berbagi
Pengetahuan}\label{c.-manfaat-peta-pengetahuan-untuk-pembelajaran-mendalam-dan-berbagi-pengetahuan}

Peta pengetahuan menawarkan berbagai manfaat pedagogis:

\begin{itemize}
\item
  \textbf{Mendorong Pembelajaran Aktif:} Proses pembuatan peta itu
  sendiri merupakan aktivitas kognitif yang mendalam, mendorong
  pemikiran kritis dan keterampilan pemecahan masalah.45
\item
  \textbf{Menciptakan ``Produk Bernilai Abadi'':} Salah satu aspek
  paling inovatif dari sistem ini adalah pemanfaatan karya mahasiswa
  sebagai bahan ajar untuk angkatan berikutnya. Dokumen laporan yang
  ``dibeli'' akan diunggah ke website kuliah, menjadi sumber belajar
  yang otentik dan relevan. Ini tidak hanya memberikan ``sense of value
  creation'' kepada mahasiswa yang membuatnya, tetapi juga menumbuhkan
  rasa kepemilikan dan kebanggaan terhadap kontribusi mereka.2
\item
  \textbf{Meningkatkan Kualitas Pembelajaran:} Dengan menyediakan sumber
  belajar yang relevan dan kontekstual yang dibuat oleh rekan sejawat,
  kualitas pembelajaran secara keseluruhan dapat meningkat. Mahasiswa
  dapat belajar dari berbagai perspektif dan strategi pemecahan masalah
  yang disajikan oleh teman-teman mereka.
\end{itemize}

Berikut adalah tabel yang memetakan tingkat Taksonomi Bloom dengan jenis
peta pengetahuan dan mata uang digital yang diusulkan:

\textbf{Tabel 1: Pemetaan Tingkat Taksonomi Bloom dengan Jenis Peta
Pengetahuan dan Mata Uang Digital}

\begin{longtable}[]{@{}
  >{\raggedright\arraybackslash}p{(\columnwidth - 6\tabcolsep) * \real{0.2500}}
  >{\raggedright\arraybackslash}p{(\columnwidth - 6\tabcolsep) * \real{0.2500}}
  >{\raggedright\arraybackslash}p{(\columnwidth - 6\tabcolsep) * \real{0.2500}}
  >{\raggedright\arraybackslash}p{(\columnwidth - 6\tabcolsep) * \real{0.2500}}@{}}
\toprule\noalign{}
\begin{minipage}[b]{\linewidth}\raggedright
Tingkat Taksonomi Bloom
\end{minipage} & \begin{minipage}[b]{\linewidth}\raggedright
Kata Kerja Kunci (Contoh)
\end{minipage} & \begin{minipage}[b]{\linewidth}\raggedright
Jenis Peta Pengetahuan
\end{minipage} & \begin{minipage}[b]{\linewidth}\raggedright
Mata Uang Digital
\end{minipage} \\
\midrule\noalign{}
\endhead
\bottomrule\noalign{}
\endlastfoot
Mengingat & Mendefinisikan, Membuat Daftar, Mengidentifikasi, Mengingat,
Menamai, Menyebutkan, Mengutip, Melaporkan, Mereproduksi, Memilih,
Menyatakan, Menabulasi, Menceritakan 4 & Peta Pengetahuan Primitif &
Point Uang \\
Memahami & Menjelaskan, Meringkas, Mengklasifikasikan, Membandingkan,
Menguraikan, Menginterpretasi, Memparafrasekan, Memprediksi, Menulis
ulang 4 & Peta Pengetahuan Primitif & Point Uang \\
Menerapkan & Menerapkan, Menghitung, Melaksanakan, Memecahkan,
Menunjukkan, Menggunakan, Mendemonstrasikan, Mengilustrasikan,
Memodifikasi, Mengoperasikan 4 & Peta Pengetahuan Aplikatif & Point
Emas \\
Menganalisis & Menganalisis, Memecah, Membandingkan, Membedakan,
Mengkategorikan, Mendeteksi, Mendiskriminasikan, Mengidentifikasi,
Menguraikan, Mengorganisir, Menghubungkan, Memisahkan, Menyusun struktur
4 & Peta Pengetahuan Aplikatif & Point Platinum \\
Mengevaluasi & Menilai, Mendukung, Mengkritik, Menentukan, Mengevaluasi,
Memutuskan, Menjustifikasi, Mengukur, Merekomendasikan, Meninjau,
Memilih 4 & Peta Pengetahuan Aplikatif Lanjutan & Point Platinum \\
Menciptakan & Merancang, Membangun, Mengembangkan, Merumuskan,
Menciptakan, Mengatur, Menggabungkan, Menyusun, Menghasilkan,
Menghipotesiskan, Mengintegrasikan, Menemukan, Memodifikasi,
Merencanakan, Mempersiapkan, Mengusulkan, Menata ulang, Merekonstruksi,
Menulis 4 & Peta Pengetahuan Aplikatif Lanjutan & Point Berlian \\
\end{longtable}

Tabel ini berfungsi sebagai ``kontrak'' sentral yang transparan antara
dosen dan mahasiswa. Bagi mahasiswa, tabel ini secara jelas
mendefinisikan ekspektasi: ``Jika saya ingin mendapatkan Point Berlian,
saya perlu membuat sesuatu yang baru (Level 6 Bloom), dan berikut adalah
kata kerja yang terkait dengannya.'' Ini memberikan peta jalan yang
jelas untuk tujuan pembelajaran dan insentif mereka. Bagi dosen, tabel
ini berfungsi sebagai rubrik dan panduan untuk merancang \emph{prompt}
tugas dan mengevaluasi kiriman, memastikan konsistensi dan keselarasan
dengan tujuan pedagogis. Ini mengoperasionalkan konsep abstrak Taksonomi
Bloom ke dalam hasil yang konkret dan terukur dalam kerangka ``knowledge
marketplace''.

\section{III. Mekanisme ``Knowledge Marketplace'': Insentif dan
Penilaian}\label{iii.-mekanisme-knowledge-marketplace-insentif-dan-penilaian}

\section{A. Struktur Mata Uang Digital Berjenjang dan
Kriterianya}\label{a.-struktur-mata-uang-digital-berjenjang-dan-kriterianya}

Sistem asesmen ini mengadopsi model ``Knowledge Marketplace'' yang
menggunakan mata uang digital berjenjang untuk memberikan insentif
kepada mahasiswa. Sistem poin ini secara langsung dihubungkan dengan
tingkat Taksonomi Bloom, memberikan kerangka insentif yang jelas dan
terstruktur untuk setiap level pencapaian kognitif.

\begin{itemize}
\item
  \textbf{Point Uang (Level 1-2 Bloom):} Diberikan untuk peta
  pengetahuan primitif yang secara akurat mendefinisikan dan menjelaskan
  konsep dasar Sinyal dan Sistem. Ini mencakup kemampuan mengingat fakta
  dan memahami ide-ide fundamental.
\item
  \textbf{Point Emas (Level 3 Bloom):} Diberikan untuk peta pengetahuan
  aplikatif yang menunjukkan kemampuan menerapkan konsep untuk
  memecahkan masalah standar atau melakukan komputasi dasar. Ini menguji
  kemampuan mahasiswa untuk menggunakan pengetahuan mereka dalam
  skenario yang dikenal.
\item
  \textbf{Point Platinum (Level 4-5 Bloom):} Diberikan untuk peta
  pengetahuan aplikatif yang menunjukkan kemampuan menganalisis sistem,
  membandingkan metode, atau merancang solusi pada tingkat menengah. Ini
  mendorong pemikiran kritis dan pemecahan masalah yang lebih kompleks.
\item
  \textbf{Point Berlian (Level 6 Bloom):} Diberikan untuk peta
  pengetahuan aplikatif lanjutan yang menunjukkan kemampuan menciptakan
  solusi orisinal, mensintesis konsep kompleks, atau mengevaluasi secara
  kritis. Ini adalah level tertinggi yang menghargai inovasi dan
  kontribusi pengetahuan yang signifikan.
\end{itemize}

Sistem ``iklan dicari karya pengetahuan'' mingguan menciptakan rasa
urgensi dan kompetisi yang sehat di antara mahasiswa. Proses pembelian
dokumen laporan oleh dosen memberikan pengakuan langsung atas upaya dan
kualitas karya mahasiswa. Selain itu, janji bahwa karya mahasiswa akan
diunggah ke ``website kuliah untuk dipelajari mahasiswa tahun depan''
memberikan rasa penciptaan nilai dan pengakuan yang kuat kepada
mahasiswa.2 Kombinasi insentif ekstrinsik (poin/mata uang) dan intrinsik
(pengakuan, kontribusi nyata) ini menciptakan lingkaran umpan balik
positif yang mendorong mahasiswa untuk menghasilkan karya berkualitas
tinggi. Mahasiswa tidak hanya belajar untuk nilai, tetapi juga untuk
kontribusi dan warisan pengetahuan mereka. Gamifikasi, terutama melalui
sistem poin dan pengakuan, telah terbukti meningkatkan motivasi dan
keterlibatan mahasiswa dalam pendidikan tinggi.46

\section{B. Sistem Pembayaran Berdasarkan Domain Sinyal dan
Sistem}\label{b.-sistem-pembayaran-berdasarkan-domain-sinyal-dan-sistem}

Sistem ini memperkenalkan dimensi insentif tambahan dengan mengaitkan
mata uang fiat yang berbeda dengan domain teknis spesifik dalam Sinyal
dan Sistem. Ini tidak hanya menambah elemen gamifikasi, tetapi juga
secara halus membimbing mahasiswa menuju area-area yang dianggap
memiliki nilai atau kompleksitas lebih tinggi dalam kurikulum.

Berikut adalah matriks pembayaran mata uang fiat yang diusulkan:

\textbf{Tabel 2: Matriks Pembayaran Mata Uang Fiat Berdasarkan Domain
Sinyal dan Sistem dan Tingkat Bloom}

\begin{longtable}[]{@{}
  >{\raggedright\arraybackslash}p{(\columnwidth - 18\tabcolsep) * \real{0.1000}}
  >{\raggedright\arraybackslash}p{(\columnwidth - 18\tabcolsep) * \real{0.1000}}
  >{\raggedright\arraybackslash}p{(\columnwidth - 18\tabcolsep) * \real{0.1000}}
  >{\raggedright\arraybackslash}p{(\columnwidth - 18\tabcolsep) * \real{0.1000}}
  >{\raggedright\arraybackslash}p{(\columnwidth - 18\tabcolsep) * \real{0.1000}}
  >{\raggedright\arraybackslash}p{(\columnwidth - 18\tabcolsep) * \real{0.1000}}
  >{\raggedright\arraybackslash}p{(\columnwidth - 18\tabcolsep) * \real{0.1000}}
  >{\raggedright\arraybackslash}p{(\columnwidth - 18\tabcolsep) * \real{0.1000}}
  >{\raggedright\arraybackslash}p{(\columnwidth - 18\tabcolsep) * \real{0.1000}}
  >{\raggedright\arraybackslash}p{(\columnwidth - 18\tabcolsep) * \real{0.1000}}@{}}
\toprule\noalign{}
\begin{minipage}[b]{\linewidth}\raggedright
Tingkat Bloom
\end{minipage} & \begin{minipage}[b]{\linewidth}\raggedright
Waktu Kontinu (WK)
\end{minipage} & \begin{minipage}[b]{\linewidth}\raggedright
Waktu Digital (WD)
\end{minipage} & \begin{minipage}[b]{\linewidth}\raggedright
Frekuensi WK
\end{minipage} & \begin{minipage}[b]{\linewidth}\raggedright
Frekuensi WD
\end{minipage} & \begin{minipage}[b]{\linewidth}\raggedright
Laplace
\end{minipage} & \begin{minipage}[b]{\linewidth}\raggedright
Z-Transform
\end{minipage} & \begin{minipage}[b]{\linewidth}\raggedright
Deret Fourier (Kontinu/Digital)
\end{minipage} & \begin{minipage}[b]{\linewidth}\raggedright
Sampling \& Rekonstruksi (WD-WK)
\end{minipage} & \begin{minipage}[b]{\linewidth}\raggedright
Konvolusi \& Dekonvolusi
\end{minipage} \\
\midrule\noalign{}
\endhead
\bottomrule\noalign{}
\endlastfoot
Level 1-2 & IDR & MYR & USD & AUD & GBP & EUR & JPY & CNY & IDR \\
Level 3 & IDR & MYR & USD & AUD & GBP & EUR & JPY & CNY & MYR \\
Level 4-5 & MYR & USD & AUD & GBP & EUR & JPY & KRW & CNY & USD \\
Level 6 & USD & AUD & GBP & EUR & JPY & KRW & CNY & JPY & AUD \\
\end{longtable}

Domain-domain yang berbeda dalam Sinyal dan Sistem (misalnya, Domain
Waktu, Domain Frekuensi, Transformasi Laplace, Z-Transform)
merepresentasikan kerangka kerja matematis dan pendekatan analitis yang
berbeda.48 Dengan menetapkan mata uang fiat yang berbeda untuk setiap
domain, dosen secara implisit memberikan nilai yang berbeda pada
pemahaman lintas domain dan keahlian khusus. Ini mendorong mahasiswa
untuk tidak hanya menguasai konsep dalam satu domain, tetapi juga untuk
membuat koneksi dan menerapkan pengetahuan di berbagai alat analitis,
yang mencerminkan pemahaman holistik tentang mata pelajaran. Misalnya,
memahami

\emph{sampling} melibatkan domain waktu diskrit (WD) dan frekuensi
diskrit (Frekuensi WD), serta interaksi di antaranya (fenomena
\emph{aliasing}).64 Diferensiasi mata uang ini dapat memotivasi
mahasiswa untuk mengeksplorasi dan menguasai area-area yang mungkin
dianggap lebih menantang atau memiliki relevansi aplikasi yang lebih
tinggi.

Penugasan mata uang fiat yang berbeda berdasarkan domain tidak hanya
menciptakan permainan, tetapi secara halus membimbing mahasiswa menuju
area-area yang dianggap memiliki nilai atau kompleksitas lebih tinggi
dalam kurikulum. Misalnya, jika Transformasi Laplace dan Z-Transform
(yang sering digunakan untuk analisis sistem dan kontrol) dibayar dalam
GBP/EUR, ini dapat menandakan sifat lanjutan atau kepentingannya di
bidang tersebut dibandingkan dengan konsep dasar domain waktu yang
dibayar IDR. Ini memberikan insentif kepada mahasiswa untuk mendalami
area-area spesifik, yang berpotensi lebih menantang, dalam Sinyal dan
Sistem yang dianggap krusial oleh dosen, secara efektif membentuk
lintasan pembelajaran mereka di luar hasil pembelajaran eksplisit. Ini
juga menciptakan dinamika di mana mahasiswa dapat berspesialisasi atau
mendiversifikasi ``portofolio pengetahuan'' mereka.

\section{C. Proses Pengajuan, Verifikasi, dan Pembelian Karya
Pengetahuan}\label{c.-proses-pengajuan-verifikasi-dan-pembelian-karya-pengetahuan}

Alur kerja sistem ``Knowledge Marketplace'' akan berjalan secara
mingguan:

\begin{enumerate}
\def\labelenumi{\arabic{enumi}.}
\item
  \textbf{Pengumuman Kebutuhan:} Setiap minggu, dosen akan membuat
  ``iklan dicari karya pengetahuan dan pemecahan masalah'' yang
  spesifik, mungkin menargetkan topik dan tingkat Bloom tertentu.
\item
  \textbf{Produksi Dokumen Laporan:} Mahasiswa akan menghasilkan dokumen
  laporan, yang idealnya berupa peta pengetahuan (primitif atau
  aplikatif) yang menjawab ``iklan'' tersebut.
\item
  \textbf{Pengajuan dan Verifikasi:} Mahasiswa menyerahkan karya mereka.
  Dosen kemudian akan melakukan verifikasi kualitas dan menilai tingkat
  Bloom yang dicapai oleh karya tersebut.
\item
  \textbf{Pembelian dan Pembayaran:} Berdasarkan penilaian, dosen akan
  ``membeli'' dokumen laporan tersebut dengan mata uang digital
  berjenjang dan mata uang fiat yang sesuai.
\item
  \textbf{Publikasi:} Dokumen yang telah ``dibeli'' akan diunggah ke
  website kuliah. Ini memastikan bahwa karya-karya terbaik mahasiswa
  menjadi sumber daya yang dapat diakses dan dipelajari oleh mahasiswa
  di tahun-tahun berikutnya, mewujudkan konsep berbagi pengetahuan.
\end{enumerate}

\section{D. Konversi Harta Terkumpul menjadi Nilai
Akhir}\label{d.-konversi-harta-terkumpul-menjadi-nilai-akhir}

Di akhir perkuliahan, total ``harta'' (akumulasi poin mata uang digital
dan mata uang fiat) yang terkumpul oleh setiap mahasiswa akan diindeks
untuk mendapatkan nilai akhir mata kuliah. Mekanisme indeksasi ini harus
didefinisikan secara transparan di awal semester, misalnya dengan
memberikan bobot tertentu untuk setiap jenis mata uang digital dan
konversi nilai tukar yang jelas untuk mata uang fiat. Hal ini memastikan
keadilan dan prediktabilitas dalam sistem penilaian.

\section{IV. Implementasi Praktis dan Rekomendasi
Teknis}\label{iv.-implementasi-praktis-dan-rekomendasi-teknis}

\section{A. Platform Digital untuk Pengelolaan dan Publikasi Karya
Mahasiswa}\label{a.-platform-digital-untuk-pengelolaan-dan-publikasi-karya-mahasiswa}

Untuk mendukung sistem ``Knowledge Marketplace'' ini, diperlukan
platform digital yang kuat dan intuitif. Platform ini akan berfungsi
sebagai pusat pengelolaan, penyimpanan, dan publikasi peta pengetahuan
yang dihasilkan mahasiswa. Fitur-fitur esensial yang harus dimiliki
platform ini meliputi:

\begin{itemize}
\item
  \textbf{Pengunggahan Dokumen Fleksibel:} Kemampuan untuk mengunggah
  berbagai format dokumen, mulai dari PDF, gambar (JPG, PNG), hingga
  format peta konsep interaktif (misalnya, dari CmapTools, MindMeister,
  atau Lucidchart).
\item
  \textbf{Kategorisasi Komprehensif:} Sistem harus memungkinkan
  kategorisasi peta berdasarkan topik materi, tingkat Taksonomi Bloom
  yang dicapai, dan domain teknis Sinyal dan Sistem yang relevan. Ini
  akan memudahkan navigasi dan pencarian.
\item
  \textbf{Fungsionalitas Pencarian Canggih:} Mahasiswa di tahun
  mendatang harus dapat dengan mudah mencari dan menemukan peta
  pengetahuan yang relevan menggunakan kata kunci, filter topik, atau
  filter tingkat Bloom.
\item
  \textbf{Interaktivitas Peta Pengetahuan:} Idealnya, platform harus
  mendukung tampilan peta pengetahuan interaktif yang memungkinkan
  pengguna untuk memperbesar, memperkecil, dan menavigasi detail dalam
  peta, seperti yang ditawarkan oleh beberapa alat pemetaan konsep
  modern.
\end{itemize}

Mahasiswa dapat memanfaatkan berbagai perangkat lunak pemetaan
konsep/mind mapping yang tersedia di pasaran untuk membuat karya mereka.
Beberapa pilihan populer yang menawarkan fitur visualisasi dan
kolaborasi yang kuat meliputi MindMeister, Lucidchart, Coggle, XMind,
SimpleMind, atau MindManager.45 Alat-alat ini memungkinkan mahasiswa
untuk fokus pada struktur dan konten pengetahuan tanpa terbebani oleh
aspek desain visual yang rumit.

Penilaian peta pengetahuan yang kompleks dari banyak mahasiswa di
berbagai tingkat Bloom dan domain akan sangat memakan waktu bagi
instruktur.1 Pemanfaatan alat bantu kecerdasan buatan (AI) dapat secara
signifikan mengotomatiskan sebagian dari proses asesmen ini. Misalnya,
AI yang digunakan untuk pembuatan

\emph{knowledge graph} atau pemahaman semantik dapat menganalisis
struktur dan konten peta pengetahuan yang diserahkan terhadap rubrik
yang telah ditentukan atau \emph{knowledge graph} ahli yang telah
dibangun sebelumnya. Sistem AI dapat mengidentifikasi konsep yang
hilang, koneksi yang lemah, atau bahkan potensi kesalahpahaman, dan
memberikan umpan balik awal kepada mahasiswa. Otomatisasi ini akan
secara signifikan mengurangi beban penilaian, memungkinkan umpan balik
yang lebih sering kepada mahasiswa, dan memungkinkan dosen untuk
berfokus pada evaluasi kualitatif tingkat tinggi dan bimbingan yang
dipersonalisasi. Ini adalah langkah penting untuk menskalakan sistem
asesmen inovatif ini dan membuatnya berkelanjutan, sejalan dengan tren
pembelajaran personalisasi berbasis AI yang meningkatkan efisiensi dan
keterlibatan.

\section{B. Pemanfaatan Alat Bantu AI untuk Pembuatan dan Validasi Peta
Pengetahuan}\label{b.-pemanfaatan-alat-bantu-ai-untuk-pembuatan-dan-validasi-peta-pengetahuan}

Integrasi AI tidak hanya terbatas pada asesmen, tetapi juga dapat
mendukung proses pembuatan dan validasi peta pengetahuan oleh mahasiswa:

\begin{itemize}
\item
  \textbf{Bantuan Pembuatan Awal:} Alat bantu AI dapat membantu
  mahasiswa dalam menghasilkan peta ``primitif'' awal. Misalnya,
  platform seperti NotebookLM atau Penseum dapat menganalisis materi
  sumber (PDF, catatan kuliah, video, atau file audio) dan menghasilkan
  ringkasan, poin-poin penting, atau pertanyaan-pertanyaan yang relevan.
  Output ini dapat menjadi dasar bagi mahasiswa untuk mulai membangun
  peta pengetahuan mereka, mempercepat proses awal dan memungkinkan
  mereka fokus pada pemahaman dan elaborasi.
\item
  \textbf{Validasi dan Koreksi:} AI dapat digunakan untuk membantu
  memvalidasi kebenaran atau kelengkapan peta ``aplikatif'' dengan
  membandingkannya dengan solusi yang diketahui atau kerangka kerja
  teoretis. Contohnya, dalam domain frekuensi, AI dapat membantu
  mengidentifikasi efek \emph{aliasing} pada representasi spektrum
  sinyal yang di-\emph{sampling} jika frekuensi sampling tidak memenuhi
  kriteria Nyquist.64 Ini memberikan lapisan verifikasi tambahan dan
  membantu mahasiswa mengidentifikasi area yang memerlukan perbaikan.
\end{itemize}

\section{C. Strategi Pedagogis untuk Mendorong Keterlibatan dan Kualitas
Karya}\label{c.-strategi-pedagogis-untuk-mendorong-keterlibatan-dan-kualitas-karya}

Keberhasilan sistem ini sangat bergantung pada strategi pedagogis yang
menyertainya:

\begin{itemize}
\item
  \textbf{Panduan yang Jelas dan Contoh Konkret:} Dosen harus
  menyediakan instruksi yang sangat jelas mengenai ekspektasi untuk
  setiap jenis peta pengetahuan dan tingkat Bloom. Menyertakan
  contoh-contoh peta pengetahuan yang baik untuk setiap level dan domain
  teknis akan sangat membantu mahasiswa dalam memahami standar kualitas.
\item
  \textbf{Umpan Balik Iteratif dan Konstruktif:} Mendorong penyempurnaan
  peta secara iteratif berdasarkan umpan balik dosen dan rekan sejawat
  adalah kunci. Ini memungkinkan mahasiswa untuk belajar dari kesalahan
  mereka dan secara bertahap meningkatkan kualitas karya mereka.
\item
  \textbf{Pembelajaran Kolaboratif:} Mendorong mahasiswa untuk bekerja
  sama dalam kelompok kecil untuk membuat dan saling mengkritik peta
  pengetahuan mereka dapat memperdalam pemahaman kolektif dan
  mengembangkan keterampilan kolaborasi.
\item
  \textbf{Pertanyaan Pemandu (Focus Questions):} Menggunakan ``focus
  questions'' atau pertanyaan pemandu yang dirancang dengan cermat dapat
  mengarahkan mahasiswa untuk menghasilkan peta yang berorientasi pada
  pemecahan masalah dan mendorong pemikiran tingkat tinggi.107
\end{itemize}

Sistem yang diusulkan, dengan ``iklan'' dan ``pembelian'' mingguan,
mengubah asesmen dari peristiwa sumatif berisiko tinggi menjadi
aktivitas pembelajaran yang berkelanjutan, berisiko rendah, dan
formatif.1 Ini mendorong keterlibatan yang konsisten dan memungkinkan
mahasiswa untuk menerima umpan balik yang sering, mengulang pemahaman
mereka, dan membangun pengetahuan mereka secara bertahap. Pergeseran ini
selaras dengan pendekatan pedagogis modern yang memprioritaskan
pengembangan berkelanjutan dan penguasaan daripada evaluasi tunggal
bertekanan tinggi, yang pada akhirnya mengarah pada pembelajaran yang
lebih mendalam dan retensi jangka panjang yang lebih baik.

Berikut adalah contoh rubrik penilaian peta pengetahuan yang dapat
digunakan:

\textbf{Tabel 3: Contoh Rubrik Penilaian Peta Pengetahuan (Primitif \&
Aplikatif) Berdasarkan Taksonomi Bloom}

\begin{longtable}[]{@{}
  >{\raggedright\arraybackslash}p{(\columnwidth - 8\tabcolsep) * \real{0.2000}}
  >{\raggedright\arraybackslash}p{(\columnwidth - 8\tabcolsep) * \real{0.2000}}
  >{\raggedright\arraybackslash}p{(\columnwidth - 8\tabcolsep) * \real{0.2000}}
  >{\raggedright\arraybackslash}p{(\columnwidth - 8\tabcolsep) * \real{0.2000}}
  >{\raggedright\arraybackslash}p{(\columnwidth - 8\tabcolsep) * \real{0.2000}}@{}}
\toprule\noalign{}
\begin{minipage}[b]{\linewidth}\raggedright
Kriteria Penilaian
\end{minipage} & \begin{minipage}[b]{\linewidth}\raggedright
Sangat Baik (Excellent)
\end{minipage} & \begin{minipage}[b]{\linewidth}\raggedright
Baik (Good)
\end{minipage} & \begin{minipage}[b]{\linewidth}\raggedright
Cukup (Fair)
\end{minipage} & \begin{minipage}[b]{\linewidth}\raggedright
Kurang (Poor)
\end{minipage} \\
\midrule\noalign{}
\endhead
\bottomrule\noalign{}
\endlastfoot
\textbf{1. Kedalaman Konsep (Bloom Level 1-6)} & Mencakup semua definisi
dan karakteristik dasar yang relevan dengan sangat jelas dan ringkas,
serta menunjukkan pemahaman mendalam tentang implikasi dan batasan
konsep. 4 & Mencakup sebagian besar definisi dan karakteristik penting,
dengan penjelasan yang memadai. 4 & Beberapa konsep penting hilang atau
dijelaskan secara tidak lengkap/kurang akurat. 4 & Banyak konsep penting
hilang atau terdapat kesalahpahaman yang signifikan. 4 \\
\textbf{2. Keterhubungan (Bloom Level 3-6)} & Menunjukkan hubungan yang
kuat, logis, dan relevan antar semua konsep utama, termasuk
\emph{cross-links} yang bermakna, mendukung pemahaman alur sistematis.
36 & Menunjukkan sebagian besar hubungan antar konsep, namun mungkin ada
beberapa koneksi yang kurang jelas atau terlewat. 36 & Beberapa konsep
terhubung, tetapi banyak hubungan penting yang hilang atau tidak logis.
36 & Konsep-konsep disajikan secara terisolasi tanpa menunjukkan
hubungan yang jelas. 36 \\
\textbf{3. Akurasi Konten} & Semua informasi teknis dan matematis
disajikan dengan akurat dan tanpa kesalahan. 110 & Sebagian besar
informasi akurat dengan beberapa kesalahan minor yang tidak signifikan.
110 & Terdapat beberapa kesalahan atau ketidakakuratan yang dapat
menyesatkan. 110 & Banyak kesalahan atau ketidakakuratan yang serius,
menunjukkan kesalahpahaman mendasar. 110 \\
\textbf{4. Struktur/Organisasi (Peta Pengetahuan Prosedural)} & Struktur
peta (misalnya, \emph{flowchart} atau \emph{diagram alir}) sangat logis,
jelas, dan mudah diikuti, merepresentasikan alur pemecahan masalah yang
efisien. 29 & Struktur peta cukup jelas, namun mungkin ada sedikit
ambiguitas dalam alur atau penempatan elemen. 29 & Struktur peta kurang
jelas atau tidak konsisten, menyulitkan pemahaman alur. 29 & Struktur
peta sangat kacau atau tidak ada, tidak dapat merepresentasikan proses
dengan baik. 29 \\
\textbf{5. Orisinalitas/Inovasi (Bloom Level 6)} & Menyajikan pendekatan
baru, sintesis konsep yang inovatif, atau solusi orisinal yang melampaui
ekspektasi, menunjukkan pemikiran tingkat tinggi. 4 & Menunjukkan
beberapa elemen kreatif atau modifikasi yang cerdas dari konsep yang
ada. 4 & Sedikit atau tidak ada elemen orisinalitas yang terlihat; hanya
reproduksi standar. 4 & Tidak ada orisinalitas atau inovasi. 4 \\
\textbf{6. Keterbacaan \& Kejelasan Visual} & Peta disajikan dengan
sangat rapi, jelas, dan estetis; penggunaan simbol, warna, dan tata
letak mendukung pemahaman optimal. 109 & Peta cukup rapi dan jelas,
dengan sedikit perbaikan visual yang dapat meningkatkan keterbacaan. 109
& Peta agak berantakan atau kurang jelas, dengan beberapa elemen visual
yang mengganggu pemahaman. 109 & Peta sangat sulit dibaca atau dipahami
karena tata letak yang buruk atau kurangnya kejelasan visual. 109 \\
\end{longtable}

Rubrik ini sangat penting untuk memastikan asesmen yang adil, konsisten,
dan transparan. Bagi mahasiswa, ini menghilangkan misteri dari proses
penilaian dengan secara jelas menguraikan apa yang merupakan pekerjaan
berkualitas pada setiap tingkat Taksonomi Bloom dan untuk setiap jenis
peta pengetahuan. Hal ini memungkinkan mereka untuk melakukan penilaian
diri dan memfokuskan upaya mereka untuk memenuhi kriteria tertentu. Bagi
dosen, ini menyederhanakan proses evaluasi, mengurangi subjektivitas,
dan menyediakan kerangka kerja standar untuk memberikan umpan balik yang
konstruktif. Ini juga membantu dalam menjaga kualitas ``basis
pengetahuan'' yang akan dibagikan kepada mahasiswa masa depan, karena
hanya peta yang memenuhi ambang kualitas tertentu yang akan ``dibeli''
dengan nilai yang lebih tinggi.

\section{V. Kesimpulan dan Prospek Masa
Depan}\label{v.-kesimpulan-dan-prospek-masa-depan}

\section{A. Rekapitulasi Manfaat
Sistem}\label{a.-rekapitulasi-manfaat-sistem}

Sistem asesmen berbasis nilai pengetahuan yang diusulkan ini menawarkan
pendekatan revolusioner untuk pembelajaran di bidang Sinyal dan Sistem.
Dengan mengintegrasikan Taksonomi Bloom, peta pengetahuan, dan mekanisme
``Knowledge Marketplace'' yang digamifikasi, sistem ini berpotensi:

\begin{itemize}
\item
  Menciptakan lingkungan pembelajaran yang sangat memotivasi dan
  menarik, mendorong partisipasi aktif dan rasa kepemilikan mahasiswa
  terhadap pembelajaran mereka.
\item
  Mendorong pemahaman mendalam dan penguasaan konsep di berbagai tingkat
  kognitif, dari mengingat dasar hingga menciptakan solusi inovatif.
\item
  Membangun komunitas pembelajar yang saling berbagi dan berkontribusi
  pada pengetahuan, menciptakan siklus pembelajaran yang berkelanjutan
  dan dinamis.
\item
  Menghasilkan artefak pembelajaran yang berharga dan berkelanjutan,
  yang dapat dimanfaatkan oleh generasi mahasiswa berikutnya.
\end{itemize}

\section{B. Tantangan dan Arah Pengembangan
Lanjutan}\label{b.-tantangan-dan-arah-pengembangan-lanjutan}

Meskipun memiliki potensi besar, implementasi sistem ini juga akan
menghadapi beberapa tantangan yang memerlukan pertimbangan dan
pengembangan lebih lanjut:

\begin{itemize}
\item
  \textbf{Skalabilitas Asesmen:} Dengan bertambahnya jumlah mahasiswa
  dan kompleksitas karya, proses penilaian secara manual akan sangat
  memakan waktu. Oleh karena itu, integrasi AI untuk otomatisasi
  sebagian proses asesmen, seperti analisis struktur peta dan
  identifikasi kesenjangan pengetahuan, akan menjadi kunci untuk menjaga
  efisiensi dan memungkinkan umpan balik yang lebih sering.
\item
  \textbf{Jaminan Kualitas Konten:} Penting untuk mengembangkan
  mekanisme yang kuat guna memastikan akurasi, relevansi, dan kualitas
  peta pengetahuan yang diunggah. Ini mungkin melibatkan proses
  \emph{peer review}, validasi oleh asisten dosen, atau pengembangan
  algoritma AI yang lebih canggih untuk verifikasi konten.
\item
  \textbf{Pencegahan Kecurangan:} Seperti sistem asesmen lainnya,
  potensi penyalahgunaan (misalnya, plagiarisme atau ketergantungan
  berlebihan pada AI tanpa pemahaman mendalam) harus dimitigasi melalui
  desain tugas yang cermat, deteksi plagiarisme, dan pendidikan etika
  akademik.
\item
  \textbf{Pengembangan Platform:} Membangun atau mengadaptasi platform
  digital yang mampu mendukung semua fitur yang diusulkan, termasuk
  integrasi mata uang digital, sistem pembayaran fiat, dan indeksasi
  nilai, akan menjadi proyek yang signifikan. Platform ini harus
  user-friendly, aman, dan dapat diandalkan.
\end{itemize}

Dengan mengatasi tantangan-tantangan ini secara proaktif dan terus
mengembangkan sistem berdasarkan umpan balik dan data, ``Knowledge
Marketplace'' ini dapat menjadi model asesmen yang transformatif, tidak
hanya meningkatkan pencapaian belajar tetapi juga menumbuhkan budaya
inovasi dan kolaborasi di kalangan mahasiswa teknik.

\bookmarksetup{startatroot}

\chapter{Journal}\label{journal}

Tentu, gagasan membuat jurnal belajar mingguan sangat bagus. Ini akan
melengkapi sistem asesmen Anda dengan memberikan wawasan tentang proses
metakognitif mahasiswa---bagaimana mereka berpikir tentang belajar itu
sendiri. Menggunakan Git atau GitHub adalah cara yang sangat cerdas
untuk melakukannya, karena ia menawarkan fitur pelacakan, kolaborasi,
dan sejarah versi yang sangat sesuai dengan konsep jurnal ini.

Berikut adalah konsep jurnal belajar mahasiswa yang memanfaatkan
Git/GitHub.

\section{Konsep ``Jurnal Belajar'' dengan
Git/GitHub}\label{konsep-jurnal-belajar-dengan-gitgithub}

Jurnal belajar ini bukan sekadar buku harian, melainkan sebuah artefak
digital yang mencatat perjalanan mahasiswa dalam menciptakan karya
pengetahuan. Setiap entri mingguan akan berfungsi sebagai ``komit''
dalam repositori Git.

\begin{center}\rule{0.5\linewidth}{0.5pt}\end{center}

\section{Struktur Repositori dan
Jurnal}\label{struktur-repositori-dan-jurnal}

Setiap mahasiswa akan memiliki repositori GitHub pribadi untuk mata
kuliah ini (atau Anda dapat menyediakan sebuah template repositori).
Strukturnya bisa sederhana:

\begin{itemize}
\item
  \textbf{\texttt{README.md}}: Halaman utama repositori yang berisi
  panduan dan tautan ke entri jurnal mingguan.
\item
  \textbf{\texttt{jurnal\_belajar/}}: Sebuah direktori yang berisi file
  Markdown untuk setiap jurnal mingguan, misalnya \texttt{minggu\_1.md},
  \texttt{minggu\_2.md}, dan seterusnya.
\end{itemize}

Setiap file Markdown mingguan (\texttt{minggu\_1.md}, misalnya) akan
memiliki struktur yang seragam:

Markdown

\begin{verbatim}
# Jurnal Belajar - Minggu 1

**Tanggal:** [Tanggal Entri]
**Topik/Karya:** [Deskripsi singkat karya pengetahuan yang sedang dibuat]

---

## Perjuangan dan Tantangan (Masalah)
* [Deskripsikan masalah yang Anda temui. Misalnya: "Saya kesulitan memahami hubungan antara deret Fourier dan transformasi Fourier."]
* [Tantangan teknis, konseptual, atau lain-lain.]

---

## Terobosan dan Solusi (Solusi)
* [Jelaskan momen "aha!" atau solusi yang Anda temukan. Misalnya: "Setelah menonton video lain, saya menyadari bahwa transformasi Fourier adalah representasi deret Fourier untuk sinyal aperiodik dengan periode yang tak terhingga."]
* [Bagaimana Anda mengatasi masalah yang dicatat di atas?]

---

## Tools dan Sumber Daya yang Digunakan (Alat)
* [Cantumkan semua alat yang Anda gunakan. Contoh: "Notebook Python untuk simulasi, YouTube untuk visualisasi konsep, diagram di buku teks."]

---

## Lesson Learned (Pembelajaran)
* [Ringkas pembelajaran utama yang Anda dapatkan dari proses ini. Ini adalah kesimpulan Anda.]
* [Apa yang akan Anda lakukan secara berbeda di minggu depan?]
\end{verbatim}

\begin{center}\rule{0.5\linewidth}{0.5pt}\end{center}

\section{Alasan Menggunakan
Git/GitHub}\label{alasan-menggunakan-gitgithub}

Memanfaatkan Git dan GitHub untuk jurnal ini memberikan beberapa manfaat
utama:

\begin{enumerate}
\def\labelenumi{\arabic{enumi}.}
\item
  \textbf{Pelacakan Sejarah (Versi Kontrol):} Setiap ``komit'' yang
  dibuat oleh mahasiswa adalah sebuah snapshot dari pemikiran mereka
  pada waktu tertentu. Ini memungkinkan Anda dan mahasiswa untuk melacak
  evolusi pemahaman mereka dari waktu ke waktu. Mahasiswa bisa melihat
  bagaimana pemahaman mereka berkembang dari ``kesulitan'' menjadi
  ``solusi'' di akhir semester.
\item
  \textbf{Transparansi dan Akuntabilitas:} Semua komit tercatat dengan
  cap waktu. Ini memastikan bahwa mahasiswa secara konsisten
  merefleksikan proses belajar mereka. Anda dapat melihat frekuensi dan
  kualitas refleksi mereka.
\item
  \textbf{Keahlian Teknis:} Mahasiswa secara tidak langsung akan belajar
  menggunakan alat Git yang sangat relevan dalam industri rekayasa
  perangkat lunak dan data. Ini memberikan mereka keterampilan tambahan
  yang berharga.
\item
  \textbf{Repositori Publik/Privat:} Repositori ini dapat disetel
  sebagai publik (untuk berbagi dengan angkatan berikutnya jika
  diizinkan) atau privat. Ini memberi fleksibilitas terkait privasi
  mahasiswa.
\end{enumerate}

\begin{center}\rule{0.5\linewidth}{0.5pt}\end{center}

\section{Cara Menilai Jurnal Belajar}\label{cara-menilai-jurnal-belajar}

Jurnal ini dapat menjadi bukti kuantitatif untuk \textbf{tingkat
Taksonomi Bloom yang lebih tinggi}, terutama \textbf{Level 5
(Mengevaluasi)} dan \textbf{Level 6 (Menciptakan)}. Anda bisa memberikan
poin penilaian berdasarkan:

\begin{itemize}
\item
  \textbf{Konsistensi:} Seberapa sering mahasiswa melakukan entri.
\item
  \textbf{Kualitas Refleksi:} Seberapa dalam mereka menganalisis
  tantangan dan terobosan. Apakah mereka hanya mencatat, atau
  benar-benar merenungkan prosesnya?
\item
  \textbf{Kejelasan ``Lesson Learned'':} Apakah kesimpulan mereka
  ringkas, jelas, dan menunjukkan pemahaman yang kuat?
\end{itemize}

Jurnal ini akan menjadi pelengkap yang sempurna untuk ``Knowledge
Marketplace'' Anda, karena ia tidak hanya mengukur produk akhir (peta
pengetahuan) tetapi juga memvalidasi proses intelektual yang
mendasarinya.

\bookmarksetup{startatroot}

\chapter{Summary}\label{summary}

In summary, this book has no content whatsoever.

\bookmarksetup{startatroot}

\chapter*{References}\label{references}
\addcontentsline{toc}{chapter}{References}

\markboth{References}{References}

\phantomsection\label{refs}
\begin{CSLReferences}{0}{1}
\end{CSLReferences}




\end{document}
